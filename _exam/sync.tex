%%% Local Variables:
%%% mode: xelatex
%%% TeX-master: "main"
%%% End:

\section*{Concorrência e Sincronização}

\exercise~As threads 0 ($T_0$) e 1 ($T_1$) acessam a variável global
$i$ com concorrência, incrementando seu valor com a expressão {\tt
  i++} da linguagem C. Supondo que o valor inicial de $i$ seja $8$, o
valor final de $i$ após a execução das 2 threads deve ser $10$. Se a
conversão da expressão para uma liguagem de montagem hipotética gerar
o seguinte código:

\begin{verbatim}
        LOAD      R1,#i      # 0. Lê o contéudo de i e escreve no registrador R1.
        ADD       R1,R1,1    # 1. Incrementa o valor contido em R1.
        STORE     R1,#i      # 2. Escreve o conteúdo de R1 em i.
\end{verbatim}

\noindent e tomarmos a convenção de que a tupla $\<T_0,1>$ indica
a execução da linha 1 da thread 0, calcule o valor final de $i$ após a
execução das seguintes sequências de instruções:

\begin{enumerate}[a)]
\item $\<T_0,0> \rightarrow \<T_0,1> \rightarrow \<T_0,2>
\rightarrow\<T_1,0>\rightarrow\<T_1,1> \rightarrow\<T_1,2>$
\item $\<T_0,0> \rightarrow\<T_1,0> \rightarrow\<T_0,1>
\rightarrow\<T_1,1>\rightarrow\<T_0,2> \rightarrow\<T_1,2>$
\item $\<T_0,0>\rightarrow\<T_0,1> \rightarrow\<T_1,0>
\rightarrow\<T_1,1>\rightarrow\<T_0,2> \rightarrow\<T_1,2>$
\item $\<T_1,0>\rightarrow\<T_1,1> \rightarrow\<T_0,0>
\rightarrow\<T_0,1>\rightarrow\<T_0,2> \rightarrow\<T_1,2>$
\item $\<T_1,0>\rightarrow\<T_1,1>\rightarrow\<T_1,2>
\rightarrow\<T_0,0>\rightarrow\<T_0,1>\rightarrow\<T_0,2>$
\end{enumerate}

\noindent Quais das sequências enumeradas garante que o valor final de $i$ seja
íntegro? O que pode ser feito para garantir que o valor final de $i$
seja igual ao valor esperado?

\exercise~As primitivas de exclusão mútua podem ser implementadas com
espera ociosa ou com bloqueio. Discuta a aplicabilidade e os méritos
relativos de cada abordagem.

\exercise~Explique como a habilitação e a desabilitação de
interrupções são úteis na implementação de primitivas de exclusão
mútua em sistemas com um único processador.

\exercise~Verifique se a implementação de {\em spin lock} usando a
função {\tt sincroniza0()} a seguir é confiável para a sincronização
das {\em threads} 0 e 1. Explique detalhadamente a verificação.\bigskip

\begin{minipage}{.48\textwidth}
\begin{lstlisting}
  /* Thread 0 */
  int haBloqueio;
  void sincroniza0() {
    while (haBloqueio == 1) {/* spin */};
    haBloqueio = 1;
    /* região crítica */
    haBloqueio = 0;
  }
\end{lstlisting}
\end{minipage}
\begin{minipage}{.48\textwidth}
\begin{lstlisting}
  /* Thread 1 */
  int haBloqueio;
  void sincroniza0() {
    while (haBloqueio == 1) {/* spin */};
    haBloqueio = 1;
    /* região crítica */
    haBloqueio = 0;
  }
\end{lstlisting}
\end{minipage}

\exercise~Verifique se a implementação de {\em spin lock} usando a
função {\tt sincroniza1(int tid)} a seguir é confiável para a
sincronização das {\em threads} 0 e 1. Explique detalhadamente a
verificação.\bigskip

\begin{minipage}{.48\textwidth}
\begin{lstlisting}
  /* Thread 0 */
  int vez;
  void sincroniza1(int tid) {/* passado valor 0 */
    /* tid é a identificação da thread */
    while (vez == 1-tid) {/* spin */};
    vez = tid;
    /* região crítica */
    vez = 1  - tid;
  }
\end{lstlisting}
\end{minipage}
\begin{minipage}{.48\textwidth}
\begin{lstlisting}
  /* Thread 1 */
  int vez;
  void sincroniza1(int tid) {/* passado valor 1 */
    /* tid é a identificação da thread */
    while (pid == 1 - tid) {/* spin */};
    vez = tid;
    /* região crítica */
    vez = 1 - tid;
  }
\end{lstlisting}
\end{minipage}

\exercise~Mostre que o algoritmo de Peterson limita o tempo de cada
thread com justiça, ou seja, uma thread não pode ser atrasada
indefinidamente em qualquer condição de espera. Em particular, mostre
que qualquer thread que esteja a espera para entrar em sua região
crítica será atrasada por não mais do que o tempo que a outra thread
leva para entrar e sair de sua região crítica uma vez.

% \exercise~Apresente uma análise detalhada do Algoritmo de Peterson
% para demonstrar que funciona adequadamente. Em particular, mostre se
% há ocorrência de impasse (deadlock) ou espera indefinida, e que a
% exclusão mútua é imposta com sucesso.

\exercise~Leia os artigos da Wikipédia buscando a palavra ``Dijkstra''
e pelo verbete ``Semáforo (computação)''. Verifique quais são as
referências de cada artigo e ligações externas (se houver).


\ifnum1=2

\section{Sincronização}
\subsection{Exercícios}

Em uma aplicação concorrente que controla saldo bancário em contas
correntes, dois processoscompartilham uma região de memória onde estão
armazenados os saldos dos clientes A e B. Osprocessos executam,
concorrentemente os seguintes passos: 

\begin{figure}[h]
\begin{tt}
\begin{tabular}{l|l}
Processo 1 (Cliente A) & Processo 2 (Cliente B) \\
\/\* saque em A */ & /* saque em A */ \\
1a. x \ra{} saldo\_do\_cliente\_A; & 2a. y \ra{} saldo\_do\_cliente\_A; \\
1b. x \ra{} x - 200; & 2b. y \ra{} y - 100;\\
1c. saldo\_do\_cliente\_A \ra{} x; & 2c. saldo\_do\_cliente\_A \ra{} y;\\
/* deposito em B */ & /* deposito em B */\\
1d. x \ra{} saldo\_do\_cliente\_B; & 2d. y \ra{} saldo\_do\_cliente\_B;\\
1e. x \ra{} x + 100; & 2e. y \ra{} y + 200;\\
1f. saldo\_do\_cliente\_B \ra{} x; & 2f. saldo\_do\_cliente\_B \ra{} y; \\
\end{tabular}
\end{tt}
\caption{Acesso concorrente à atualização de conta bancária.}
\end{figure}

Supondo que os valores dos saldos de A e B sejam, respectivamente, 500
e 900, antes de os processosexecutarem, pede-se:

\begin{itemize}
\item Quais os valores corretos esperados para os saldos dos clientes
  A e B após o término da execuçãodos processos?\\

R: Cliente A = 200 e Cliente B = 1.200

\item Quais os valores finais dos saldos dos clientes se a sequência
  temporal de execução das operaçõesfor: 1a, 2a, 1b, 2b, 1c, 2c, 1d,
  2d, 1e, 2e, 1f, 2f?\\

R: Cliente A = 400 e Cliente B = 1.100

\item Utilizando semáforos, proponha uma solução que garanta a
  integridade dos saldos e permita o maiorcompartilhamento possível
  dos recursos entre os processos, não esquecendo a especificação
  dainicialização dos semáforos.
\end{itemize}

\begin{figure}
\centering
\begin{tt}
\begin{tabular}{l|l}
Processo 1 (Cliente A) & Processo 2 (Cliente B)\\
/* saque em A */ & /* saque em A */\\
Down (S1) & Down (S1)\\
x \ra{} saldo\_do\_cliente\_A; & y \ra{} saldo\_do\_cliente\_A;\\
x \ra{} x - 200; & y \ra{} y - 100; \\ 
saldo\_do\_cliente\_A \ra{} x; &  saldo\_do\_cliente\_A \ra{} y;\\
Up (S1) & Up (S1)\\
/* deposito em B */ & /* deposito em B */\\
Down (S2) & Down (S2)\\
x \ra{} saldo\_do\_cliente\_B; & y \ra{} saldo\_do\_cliente\_B;\\
x \ra{} x + 100; & y \ra{} y + 200;\\
saldo\_do\_cliente\_B \ra{} x; & saldo\_do\_cliente\_B \ra{} y;\\
Up (S2) & Up (S2) \\ 
\end{tabular}
\end{tt}

\caption{Acesso concorrente utilizando semáforo para evitar inconsistências.}
\end{figure}


O problema dos leitores/escritores, apresentado a seguir, consiste em
sincronizar processos queconsultam/atualizam dados em uma base
comum. Pode haver mais de um leitor lendo ao mesmo tempo;no entanto,
enquanto um escritor está atualizando a base, nenhum outro processo
pode ter acesso a ela(nem mesmo leitores).

\begin{tt} 
\begin{tabbing}
  VAR \=Acesso: Semaforo := 1;\\
  \>Exclusao: Semaforo := 1;\\
  \>Nleitores: integer := 0;\\

PROCEDURE Escritor;\\
BEGIN\\
\> ProduzDado;\\
\> DOWN (Acesso);\\
\> Escreve;\\
\> UP (Acesso);\\
END;\\

PROCEDURE Leitor;\\
BEGIN\\
\> DOWN (Exclusao);\\
\> Nleitores := Nleitores + 1;\\
\> IF ( Nleitores = 1 ) THEN DOWN (Acesso);\\
\> UP (Exclusao);\\
\> Leitura;\\
\> DOWN (Exclusao);\\
\> Nleitores := Nleitores - 1;\\
\> IF ( Nleitores = 0 ) THEN UP (Acesso);\\
\> UP (Exclusao);\\
\> ProcessaDado;
END;\\
 
\end{tabbing}
\end{tt}
 
\begin{itemize}
 
\item Suponha que exista apenas um leitor fazendo acesso à
  base. Enquanto este processo realiza aleitura, quais os valores das
  três variáveis?  

  R: Acesso=0 Exclusao=1 Nleitores=1
 
\item Chega um escritor enquanto o leitor ainda está lendo. Quais os
  valores das três variáveis após obloqueio do escritor ? Sobre
  qual(is) semáforo(s) se dá o bloqueio?  

R: Acesso=0 Exclusao=1 Nleitores=1, o bloqueio ocorre no semáforo
Acesso.
 
\item Chega mais um leitor enquanto o primeiro ainda não acabou de ler
  e o escritor está bloqueado.Descreva os valores das três variáveis
  quando o segundo leitor inicia a leitura.  

R: Acesso=0 Exclusao=1 Nleitores=2
 
\item Os dois leitores terminam simultaneamente a leitura. É possível
  haver problemas quanto àintegridade do valor da variável nleitores?
  Justifique.  

R: Não, pois a exclusão mútua a esta variável é implementada pelo
semáforo Exclusao.
 
\item Descreva o que acontece com o escritor quando os dois leitores
  terminam suas leituras. Descrevaos valores das três variáveis quando
  o escritor inicia a escrita.  
  
  R: O processo Escritor inicia a escrita. Acesso=0 Exclusao=1
  Nleitores=0
 
\item Enquanto o escritor está atualizando a base, chagam mais um
  escritor e mais um leitor. Sobrequal(is) semáforo(s) eles ficam
  bloqueados? Descreva os valores das três variáveis após obloqueio
  dos recém-chegados.  

R: Os processo ficam bloqueados no semáforo Acesso. Acesso=0
Exclusao=0 Nleitores=1
 
\item Quando o escritor houver terminado a atualização, é possível
  prever qual dos processosbloqueados (leitor ou escritor) terá acesso
  primeiro à base?  

R: Não, em geral os sistemas operacionais utilizam a escolha randômica
dentre os processos em estado deespera.
 
\item Descreva uma situação onde os escritores sofram starvation
  (adiamento indefinido).  

  R: Caso um processo Escritor esteja aguardando, bloqueado pelo
  semáforo Acesso, e sempre surgirem novosprocessos Leitor, o processo
  Escritor pode nunca ganhar acesso ao recurso.
\end{itemize}

\end{document}
\fi


  \question[3] Dois processos acessam a variável global {\tt
    numero\_de\_acessos}, que indica o número de acessos em uma página
  web, de acordo com as instruções a seguir:
  
  \begin{tt}
  \begin{center}
    \begin{tabular}{|l|}\hline
      \bf Processo 0\\\hline
      0a. x $\leftarrow$ numero\_de\_acessos\\
      0b. x $\leftarrow$ x + 10\\
      0c. numero\_de\_acessos $\leftarrow$ x\\\hline
    \end{tabular}
  \end{center}
\end{tt}

\begin{tt}
  \begin{center}
    \begin{tabular}{|l|}\hline
      \bf Processo 1\\\hline
      1a. y $\leftarrow$ numero\_de\_acessos\\
      1b. y $\leftarrow$ y + 18\\
      1c. numero\_de\_acessos $\leftarrow$ y\\\hline
    \end{tabular}
  \end{center}
\end{tt}

\reinitrand[counter=rand,first=128,last=2048] 

Supondo que o valor inicial da variável {\tt numero\_de\_acessos} é
\rand{\arabic{rand}}, responda as seguintes questões:

\begin{parts}

\part 
Qual o valor esperado para a variável {\tt numero\_de\_acessos},
se a execução for sequencial, com o {\tt Processo 0} executando
primeiro?

\part
\label{unsync} Se o acesso não for sequencial, mas concorrente,
qual o valor final da variável {\tt numero\_de\_acessos} se a
sequência temporal de execução for {\tt 0a, 1a, 0b, 1b, 0c, 1c}?

\part Se houver inconsistência no valor esperado para {\tt
  numero\_de\_acessos} após a execução da sequência descrita no item
\ref{unsync}, como ela poderia ser consertada utilizando os
mecanismos de sincronização? Demonstre como seria o código
corrigido.

\end{parts}

  \question[1] Qual a diferença básica entre os mecanismos de
  sincronização {\bf \em spin lock} e {\em \bf mutex}? Para as
  características ou requisitos dos processos listados na tabela,
  escolha qual mecanismo é mais adequado para sincronização.
  \bigskip
  \def\whichlock{( ) {\em spin lock}\hbox{     } ( ) {\em mutex}}
  \begin{table}[h]
    \centering
    \begin{tabular}{|c|c|}\hline
      \bf Característica/Requisito & \bf Bloqueio recomendado\\\hline\hline
      Baixa sobrecarga de bloqueio & \whichlock\\\hline
      Tempo curto de bloqueio & \whichlock\\\hline
      Tempo longo de bloqueio & \whichlock\\\hline
      Necessidade de espera durante a posse do bloqueio & \whichlock\\\hline
    \end{tabular}
    \caption{Bloqueio recomendado de acordo com as características do
      bloqueio do processo.}
    \label{so:sync}
  \end{table}


\question[3] O algoritmo de Peterson pode ser usado para implementar
o mecanismo de bloqueio do tipo {\em spin} ({\em spin lock}), que evita
que duas {\em threads} ou processos interessados no mesmo recurso entrem
na região crítica ao mesmo tempo. O algoritmo em {\sc C} é listado a seguir:

\lstset{numbers=left,numberstyle=\scriptsize,
        commentstyle=\footnotesize\color{gray},
        frame=single}
\begin{lstlisting}
int vez; 
int interessado[2]; 

void Peterson(int tid) { 
     int outro; 
     outro = 1 - tid; 

     interessado[tid] = 1; 
     vez = tid; 

     while(vez == tid && interessado[outro] == 1)
                /* laço infinito */;

     /* REGIÃO CRÍTICA */

     interessado[tid] = 0;
}
\end{lstlisting}

\noindent O argumento {\tt int tid} recebe o valor 0 ou 1, indicando em qual das
2 {\em threads} o algoritmo está sendo executado.

Vamos adotar a seguinte convenção: $T_{n,m}$ indica que a {\em thread}
$n$ está executando a linha $m$ do código. Por exemplo, $T_{0,9}$
indica que a {\em thread} $0$ executa a linha $9$, ou seja, {\tt vez =
0}.

Realize as sequências a seguir, e indique qual {\em thread} entrará na
região crítica e qual {\em thread} ficará esperando no laço infinito.

\begin{enumerate}[a)]

\item $T_{0,5}$, $T_{0,6}$, $T_{0,8}$, $T_{0,9}$, $T_{1,5}$, $T_{1,6}$,
$T_{1,9}$, $T_{0,11}$, $T_{1,11}$;

\item $T_{1,5}$, $T_{1,6}$, $T_{1,8}$, $T_{1,9}$, $T_{0,5}$, $T_{0,6}$,
$T_{0,9}$, $T_{0,11}$, $T_{1,11}$;

\end{enumerate}

O que acontece se as duas {\em threads} executarem a linhas $11$ ao
mesmo tempo?
