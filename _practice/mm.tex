\maketitle

\warning

\section*{Gerenciamento de Memória}

\paragraph{1.  Espaço de endereçamento e segmentação.} Baixe o código 

\bigskip

\CMD{wget https://raw.githubusercontent.com/ajholanda/edu-os/master/mm/address-space.c}

\noindent e compile-o usando o comando

\bigskip

\CMD{gcc -g -o addr address-space.c$\quad$ \#.}

\noindent Execute-o em segundo plano e verifique o espaço de
endereçamento do processo pelos comandos:

\bigskip

\CMD{./addr \&}
\CMD{pmap <PID>}
\CMD{nm addr}

\noindent onde {\tt PID} é o indentificador do processo.

\noindent Escreva o endereço virtual, em hexadecimal e decimal, dos
seguintes elementos:

\begin{enumerate}[a)]
\item Endereço da variável global {\tt glob};
\item Endereço da variável global {\tt gloc};
\item Endereço da função {\tt main()};
\item Endereço da função {\tt loop()};
\item Endereço da variável local de {\tt loop()} chamada {\tt loc};
\item Endereço para o qual a variável global {\tt gloc} aponta.
\end{enumerate}

\hrule

\paragraph{2. Memória virtual.}~Verifique a estatística de uso da memória virtual
usando o comando 

\bigskip

\CMD{vmstat 10 5}

\noindent onde 5 é o número de estatísticas a serem geradas e 10 é o
intervalo entre elas em segundos. Obtenha os seguintes dados:

\begin{enumerate}[a)]
\item  Faça uma tabela contendo os valores
  de memória virtual usada, memória livre, memória ativa e memória
  inativa. 
\item Calcule a média e o desvio-padrão para cada campo.
\end{enumerate}

\noindent Use o comando

\bigskip

\CMD{man vmstat}

\noindent para saber o significado da saída do programa.

%% Local variables:
%% TeX-master: main
%% End: