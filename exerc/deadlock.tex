\section*{Deadlock (impasse)}

\exercise~Durante a execução de 3 processos (ou threads) $P_1, P_2,
P_3$, há 2 recursos $R_1, R_2$ a serem compartilhados. A notação
$P_i\rightarrow R_j$ indica que o processo $i$ faz requisição do
recurso $j$.  Faça o grafo de alocação de recursos e verifique se há
ocorrência de deadlock (impasse) para as seguintes requisições:

\begin{enumerate}[i)]
\item  $P_1 \rightarrow R_2,\qquad P_2 \rightarrow R_2, \qquad P_1
  \rightarrow R_1,\qquad P_2 \rightarrow R_1, \qquad P_3 \rightarrow
  R_2.$

\item  $P_2 \rightarrow R_1,\qquad P_1 \rightarrow R_1, \qquad P_2
  \rightarrow R_2,\qquad P_1 \rightarrow R_2, \qquad P_3 \rightarrow
  R_1.$
\item  $P_1 \rightarrow R_2,\qquad P_2 \rightarrow R_1, \qquad P_1
  \rightarrow R_1,\qquad P_2 \rightarrow R_2, \qquad P_3 \rightarrow
  R_2.$

\item  $P_2 \rightarrow R_2,\qquad P_2 \rightarrow R_1, \qquad P_1
  \rightarrow R_1,\qquad P_1 \rightarrow R_2, \qquad P_3 \rightarrow
  R_2.$

\item  $P_1 \rightarrow R_1,\qquad P_2 \rightarrow R_2, \qquad P_1
  \rightarrow R_2,\qquad P_2 \rightarrow R_1, \qquad P_3 \rightarrow
  R_2.$

\end{enumerate}

\exercise~Para o fragmento de código a seguir, onde 2 threads tentam
obter o bloqueio de 2 recursos, descreva as possíveis sequências de
execução, verificando se há ocorrência de impasse ({\em deadlock}).

\lstset{language=C,frame=tlr,
commentstyle=\small\color{black!85}}

\begin{lstlisting}
  /* variáveis globais de controle da exclusão mútua */
  mutex_t recurso1, recurso2; 
\end{lstlisting}
\lstset{frame=single}
\begin{minipage}{.5\textwidth}
    \begin{lstlisting}
      /*  Thread 1 */
     mutex_lock(&recurso1);
      mutex_lock(&recurso2);
      /**
      * região crítica
      */
      mutex_unlock(&recurso2);
      mutex_unlock(&recurso1);
    \end{lstlisting}
\end{minipage}
\begin{minipage}{.5\textwidth}
    \begin{lstlisting}
      /* Thread 2 */
      mutex_lock(&recurso2);
      mutex_lock(&recurso1);
      /**
      * região crítica
      */
      mutex_unlock(&recurso1);
      mutex_unlock(&recurso2);
    \end{lstlisting}
\end{minipage}


\def\DEADLOCK{
  Explique também as situações que conduzem a um {\em deadlock} entre
  processos e exemplifique utilizando um grafo de alocação de processos.
}


\question[2] Faça o grafo de alocação de recursos para as requisições
das {\em threads} listadas a seguir, informando quando houver
{\em deadlock}.

\begin{enumerate}[a)]
\item $T_0$ requisita $R_0$, $T_0$ requisita $R_1$,
  $T_1$ requisita $R_1$, $T_1$ requisita $R_0$;

\item $T_0$ requisita $R_0$, $T_1$ requisita $R_1$,
  $T_1$ requisita $R_0$, $T_0$ requisita $R_1$;

\item $T_0$ requisita $R_0$, $T_2$ requisita $R_1$,
  $T_1$ requisita $R_1$, $T_1$ requisita $R_0$;
  $T_0$ libera $R_0$, $T_2$ requisita $R_0$;
  
\item $T_0$ requisita $R_0$, $T_2$ requisita $R_1$,
  $T_1$ requisita $R_1$, $T_1$ requisita $R_0$;
  $T_0$ requisita $R_1$, $T_2$ requisita $R_0$.

\end{enumerate}
