
\section*{Gerenciamento de Entrada e Saída (E/S)}


\exercise~Há a necessidade de escalonamento das requisições provenientes
    dos dispositivos de caracter? Por quê?

\exercise~O que é e onde se localiza o {\em buffer} de dados dos
    dispositivos de E/S?

\exercise~Por quê os sistemas operacionais exigem de todos os {\em
      drivers} de dispositivo a mesma interface (protótipo de função)
    padrão?

    \exercise~Descreva se a interrupção ou acesso direto à memória
    (DMA) é mais adequado para os seguintes dispositivos:
\begin{enumerate}[a)]
\item teclado;
\item disco rígido;
\item placa de rede.
\end{enumerate}

\noindent Descreva as etapas do processo de transferência de dados
destes dispositivos com o processador, sistema operacional e
manipulador de interrupção.

\exercise~Quais são os $3$ componentes principais no tempo de acesso ao
    disco?

\exercise~Dada as seguintes sequências de requisições para um disco
    com $200$ trilhas:
    \begin{center}
      $55\ 58\ 39\ 88\ 18\ 90\ 160\ 150\ 38\ 122\ 184$
    \end{center}
    onde a posição atual dos cabeçotes é $100$, calcule a quantidade
    de movimentos dos cabeçotes para atender às requisições utilizando
    os seguintes algoritmos de escalonamento de E/S:
    
    \begin{enumerate}[a)]
    \item FCFS;
    \item SSTF;
    \item SCAN;
    \item C-SCAN.
    \end{enumerate}
 
\section*{Principais t\'opicos a serem estudados para a prova}

\begin{enumerate}
\item Dispositivos de caracter, bloco e rede;
\item Interfaces de E/S com o sistema operacional;
\item {\em Buffer};
\item Escalonamento em dispositivos de E/S.
\end{enumerate}

