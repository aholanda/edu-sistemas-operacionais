\maketitle
\section*{Sistemas de Arquivos}

\exercise~O que é o sistema de arquivos virtual dos SOs baseados em
{\sc Unix}?  Descreva as suas principais entidades, destancando a
diferença entre elas.

\exercise~Dê um exemplo de chamada de sistema que utilize o sistema
virtual para abrir um arquivo. Há diferenças entre esta chamada se o
arquivo estiver armazenado em um sistema de arquivos {\sc ntfs}, {\tt
  ext4} ou {\sc hfs} ({\em Hierarchical File System})?

\exercise~Explique como os sistemas de arquivos podem evitar que haja 
corrupção das entidades devido à ocorrência de falhas no SO.

\exercise~Como a fragmentação dos arquivos na memória secundária pode
degradar a performance do sistema de arquivos.

\exercise~O sistema de arquivos FAT32 usa 32 bits para a identificação
do {\em cluster}. Para um disco onde o tamanho de cada setor é 512
bytes, complete a tabela abaixo com os valores do tamanho padrão para
cada cluster, de acordo com o tamanho da partição para o FAT32.

\begin{center}
  \begin{tabular}[ht]{l|l}\hline
    \bf tamanho da partição & \bf tamanho padrão do {\em cluster} \\\hline
    <32MB & não suportado\\\hline
    32MB--64MB & 512 bytes\\\hline
    64MB--128MB & 1KB\\\hline
    128MB--256MB & 2KB\\\hline
    256MB--8GB & 4KB\\\hline
    8GB--16GB & 8KB\\\hline
    16GB--32GB & 16KB\\\hline
    >32GB & não suportado \\\hline
  \end{tabular}
\end{center}
\def\VFS{
\question[5,0] Responda às seguintes questões baseadas no sistemas de arquivos virtual:

\begin{enumerate}[a)]
\item O que é o sistema de arquivos virtual dos SOs baseados em {\sc
    Unix}?
\item Descreva as suas principais entidades, destancando a diferença
  entre elas.
\item Dê um exemplo de chamada de sistema que utilize o sistema
  virtual para abrir um arquivo.
\item Há diferenças entre esta chamada se o arquivo estiver armazenado
  em um sistema de arquivos {\sc ntfs}, {\tt ext4} ou {\sc hfs} ({\em
    Hierarchical File System})?
\end{enumerate}
}