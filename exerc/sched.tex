
\question[1,5] Calcule o tempo de espera médio para os processos
mostrados na tabela~\ref{tab:simple}, utilizando os seguintes
algoritmos:

\begin{enumerate}
\item Primeiro a chegar, primeiro a ser servido (FIFO -- {\em first in
  first out});
\item Processo de menor tempo primeiro (SJF -- {\em shortest job first});
\item Prioridade.
\end{enumerate}

\begin{table}[h]
  \centering
  \begin{tabular}{cccc}\hline
    \bf processo & \bf tempo & \bf prioridade \\\hline
    $p_1$ & 4 & 3 \\
    $p_2$ & 2& 2\\
    $p_3$ & 2& 1\\
    $p_4$ & 3& 2\\\hline
  \end{tabular}
  \caption{Tempo de execução e prioridade dos processos para o exercício.}
  \label{tab:simple}
\end{table}

\question[2,5] Considere dois
processos que possuem 4 ciclos, 2 de processador (CPU) e 2 de
entrada/saída  (E/S) a serem executados em um sistema, conforme mostrado na
tabela~\ref{tab:ioproc}. A E/S está relacionada à requisição de
leitura ou escrita em um único disco rígido.

\begin{table}[h]
  \centering
  \begin{tabular}{c|cc|cc}\hline
    & ciclo 1 & & ciclo 2 &  \\\hline
    \bf processo & \bf CPU & \bf disco & \bf CPU & \bf disco \\\hline
    $p_1$ & 16 & 10 & 8 & 10  \\
    $p_2$ & 3 & 30 & 5 & 15 \\\hline
  \end{tabular}
  \caption{Tabela contendo os tempos de execução de 2 processos de 2
    ciclos cada.}
  \label{tab:ioproc}
\end{table}

Utilizando o escalonamento {\em Round-Robin} com quantum de 4 unidades
de tempo, desenhe o diagrama de tempo para o escalonamento dos
processos mostrados na tabela~\ref{tab:ioproc} e calcule a taxa de
ocupação do disco rígido e da CPU.


\def\quantumtotal{11} \def\quantumA{8} \def\quantumB{3}
\def\quantumfila{2} \question[3,0] Em um sistema operacional o
escalonador utiliza {\bf \quantumfila{}} filas {\bf A} e {\bf B}. O
algoritmo entre as filas é de revezamento (Round-Robin). De cada
\quantumtotal{} unidades de tempo (quantum) do processador,
\quantumA{} são fornecidas para os processos da fila {\bf A} e
\quantumB{} para os processos da fila {\bf B}. O tempo de cada fila
também é dividido entre os processos por revezamento (Round-Robin),
com quantum de 2 unidades de tempo para todos. A tabela~\ref{so:sched}
mostra o conteúdo das duas filas no instante zero.

\begin{table}[h]
  \centering
  \begin{tabular}{c|c|c}\hline
    \bf Fila & \bf Processo & \bf Tempo de execução \\ \hline\hline
    A & $P_1$ & 6 \\\hline
    A & $P_2$ & 7 \\\hline
    A & $P_3$ & 11 \\\hline
    B & $P_4$ & 3 \\\hline
    B & $P_5$ & 8 \\\hline
    B & $P_6$ & 2 \\\hline
  \end{tabular}
  \caption{Duração em unidades de tempo do ciclo das tarefas existentes nas filas A e B.}
  \label{so:sched}
\end{table}


\noindent Considere que está se iniciando um ciclo de 11 unidades, e
agora a fila {\bf A} vai receber \quantumA{} unidades de tempo. Faça o digrama
de tempo (Gantt) dos processos, calculando o tempo médio total de espera em
cada fila e o tempo médio total considerando todos os processos.

\noindent Obs: Se terminar a fatia de tempo da fila X no meio do 
quantum de um dos processos, o processador passa para outra
fila. Entretanto, este processo permanece como primeiro na fila X até
que todo seu quantum seja consumido.
