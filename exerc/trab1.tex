
%%% Local Variables: 
%%% mode: latex
%%% TeX-master: t
%%% End: 

\documentclass[11pt,twoside]{article}

\usepackage[brazil]{babel}
\usepackage[utf8]{inputenc}
\usepackage[a4paper,left=1.5cm,top=1.5cm,right=1.5cm]{geometry}
\usepackage{enumerate}
\usepackage{graphicx}
\usepackage{tikz}


\newcounter{qno}\setcounter{qno}{0}
\newcounter{grade}\setcounter{grade}{0}
\def\newex#1{\addtocounter{qno}{1}\addtocounter{grade}{#1}
\paragraph{Questão \arabic{qno}. [#1 ponto(s)]}\marginpar{\tiny $\frac{\arabic{grade}}{10}$}}
%% MARGINPAR
\setlength{\marginparwidth}{1.2in}
\let\oldmarginpar\marginpar
\renewcommand\marginpar[1]{\-\oldmarginpar[\raggedleft\tiny#1]%
{\raggedright\tiny#1}}

\begin{document}

\title{1$^o$ Trabalho -- Sistemas Operacionais I}
\author{Prof. Adriano J. Holanda}
\date{\today}
\maketitle
{\noindent Nome: \hbox{$\underline{\hspace{10cm}}$} Número: \hbox{$\underline{\hspace{3.5cm}}$}\\}
{\noindent Curso: Sistemas de Informação -- FAFRAM -- $4^o$ Ciclo}

\newex{1} Descreva as diferenças entre a arquitetura monolítica e a baseada
em \emph{microkernel} para os sistemas operacionais

\newex{1} A partir do bloco de controle de processo a seguir:

\def\recbase{4}
\def\recheight{1}
\begin{center}
  \begin{tikzpicture}[scale=0.775,mem/.style={fill=gray!60},
    var/.style={fill=blue!30}]
    
    \draw[var] (0,2+5.5*\recheight) rectangle (\recbase,2+6.5*\recheight)
    node[midway] {\scriptsize{estado do processo}};
    \draw[var] (0,2+4.5*\recheight) rectangle (\recbase,2+5.5*\recheight)
    node[midway] {\scriptsize{número do processo}};
    \draw[var] (0,2+3.5*\recheight) rectangle (\recbase,2+4.5*\recheight)
    node[midway] {\scriptsize{contador de programa}};
    \draw[mem] (0,2+2*\recheight) rectangle (\recbase,2+3.5*\recheight) node[midway] {registradores};
    \draw[var] (0,2+\recheight) rectangle (\recbase,2+2*\recheight)
    node[midway] {\scriptsize limites de
    memória};
    \draw[var] (0,1.5) rectangle (\recbase,2+\recheight) node[midway] {\scriptsize{lista de
    arquivos abertos}};
    \draw[mem] (0,0) rectangle (4,1.5) node[midway] {$\ldots$};
  \end{tikzpicture}
\end{center}

Descreva a função dos seguintes atributos:
\begin{enumerate}
\item Estado do processo;
\item Contador do programa;
\item Registradores;
\item Número ou identificador do processo.
\end{enumerate}


\newex{1} Qual a sequência para a criação de um processo? Descreva
cada um dos estados desta sequência e a função do bloco de controle do
processo (PCB - \emph{Process Control Block}).

\newex{1} Desenhe o diagrama de estados do processo dos sistemas
operacionais modernos, descrevendo cada estado. Se um processo for
limitado por E/S (entrada e saída), ou seja utiliza mais recursos de
entrada e saída do que processamento, e este processo não estiver em
execução, qual seria o estado mais provável deste processo? E se o
processo for limitado por processamento, ou seja utiliza mais o
processador do que requisições de entrada e saída, qual o estado mais
provável quando este processo não estiver em execução?

\newex{1} Descreva quais elementos são compartilhados pelas {\em
  threads} em um processo {\em multithread} e o motivo destes
elementos serem compartilhados?

\newex{1} Calcule o tempo de espera médio para os processos
mostrados na tabela~\ref{tab:simple}, utilizando os seguintes
algoritmos:

\begin{enumerate}
\item Primeiro a chegar, primeiro a ser servido (FIFO -- {\em first in
  first out});
\item Processo de menor tempo primeiro (SJF -- {\em shortest job first});
\item Prioridade.
\end{enumerate}

\begin{table}[h]
  \centering
  \begin{tabular}{cccc}\hline
    \bf processo & \bf tempo & \bf prioridade \\\hline
    $p_1$ & 4 & 3 \\
    $p_2$ & 2& 2\\
    $p_3$ & 2& 1\\
    $p_4$ & 3& 2\\\hline
  \end{tabular}
  \caption{Tempo de execução e prioridade dos processos para o exercício~\arabic{qno}.}
  \label{tab:simple}
\end{table}

\newex{2} Em um sistema operacional o escalonador utiliza {\bf 2}
filas {\bf A} e {\bf B}. O algoritmo entre as filas é de revezamento
(Round-Robin). De cada 11 unidades de tempo (quantum) do processador,
7 são fornecidas para os processos da fila {\bf A} e 4 para os
processos da fila {\bf B}. O tempo de cada fila também é dividido
entre os processos por revezamento (Round-Robin), com quantum de 2
unidades de tempo para todos. A tabela~\ref{so:sched} mostra o
conteúdo das duas filas no instante zero.

\begin{table}[h]
  \centering
  \begin{tabular}{c|c|c}\hline
    \bf Fila & \bf Processo & \bf Tempo de execução \\ \hline\hline
    A & $P_1$ & 6 \\\hline
    A & $P_2$ & 5 \\\hline
    A & $P_3$ & 7 \\\hline
    B & $P_4$ & 3 \\\hline
    B & $P_5$ & 8 \\\hline
    B & $P_6$ & 4 \\\hline
  \end{tabular}
  \caption{Duração em unidades de tempo do ciclo das tarefas existentes nas filas A e B.}
  \label{so:sched}
\end{table}

\noindent Considere que está se iniciando um ciclo de 11 unidades, e
agora a fila {\bf A} vai receber 7 unidades de tempo. Faça o digrama
de tempo (Gantt) dos processos, calculando o tempo médio total de espera em
cada fila e o tempo médio total considerando todos os processos.

\noindent Obs: Se terminar a fatia de tempo da fila X no meio do 
quantum de um dos processos, o processador passa para outra
fila. Entretanto, este processo permanece como primeiro na fila X até
que todo seu quantum seja consumido.

\newex{2} Considere dois processos que possuem 4 ciclos, 2 de
processador (CPU) e 2 de entrada/saída (E/S) a serem executados em um
sistema, conforme mostrado na tabela~\ref{tab:ioproc}. A E/S está
relacionada à requisição de leitura ou escrita em um único disco
rígido.

\begin{table}[h]
  \centering
  \begin{tabular}{c|cc|cc}\hline
    & ciclo 1 & & ciclo 2 &  \\\hline
    \bf processo & \bf CPU & \bf disco & \bf CPU & \bf disco \\\hline
    $p_1$ & 16 & 10 & 8 & 10  \\
    $p_2$ & 3 & 30 & 5 & 15 \\\hline
  \end{tabular}
  \caption{Tabela contendo os tempos de execução de 2 processos de 2
    ciclos cada.}
  \label{tab:ioproc}
\end{table}

Utilizando o escalonamento {\em Round-Robin} com quantum de 4 unidades
de tempo, desenhe o diagrama de tempo para o escalonamento dos
processos mostrados na tabela~\ref{tab:ioproc} e calcule a taxa de
ocupação do disco rígido e da CPU.

\end{document}
