
\section*{Bloqueio e Sincronização}

\paragraph{mutex}~Baixe os códigos usando os comandos\medskip

\CMD{wget https://raw.githubusercontent.com/ajholanda/edu-os/master/sync/incr-nomutex.c\\
  \$ wget https://raw.githubusercontent.com/ajholanda/edu-os/master/sync/incr-mutex.c}

\noindent e analise

\begin{itemize}
\item como as threads foram criadas,
\item a função de {\tt pthread\_join()},
\item qual a função invocada pela thread e quais seus argumentos.
\end{itemize}

\noindent Compile os códigos usando os comandos\medskip

\CMD{gcc -pthread -o incr-nomutex incr-nomutex.c\\
  \$ gcc -pthread -o incr-mutex incr-mutex.c}

\noindent e execute-os fornecendo os seguintes valores como
parâmetros:

\begin{center}
  $\langle 100, 1.000, 10.000, 100.000, 1.000.000, 10.000.000, 100.000.000 \rangle$.
\end{center}

\noindent Verifique o tempo de execução de cada parâmetro usando o comando {\tt
  time} e compare o valor final com o esperado.

\paragraph{kernel}

Baixar o código do kernel do Linux usando o comando 

\CMD{wget https://www.kernel.org/pub/linux/kernel/v3.x/linux-3.15.1.tar.xz}

\noindent e descompactá-lo no diretório atual com o comando

\CMD{tar xfvJ linux-3.15.1.tar.xz $\qquad$\#.}

\noindent Entre no diretório contendo o código fonte do kernel e
verificque as implementações dos principais mecanismos de
sincronização usando os comandos

\CMD{cd cd linux-3.15.1/kernel/\\
  \$  cd locking/\\
  \$  less spinlock.c\\
  \$ less semaphore.c\\
  \$ less mutex.c $\qquad$\#.}
