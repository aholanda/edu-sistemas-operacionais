\section*{Processos}

Os processos possuem características variadas para cada sistema operacional,
porém, um grupo de funções foram se solidificando e praticamente ajudam a
padronizar a semântica dos elementos do processo mesmo para diferentes sistemas
operacionais. Grande parte desta padronização é atribuída à adesão à
especificação \href{http://pubs.opengroup.org/onlinepubs/9699919799/}{POSIX}.

\paragraph{Base} Explicar os seguintes conceitos, fundamentais para a operação
mínima de qualquer ambiente Unix:

\begin{itemize}
\item {\em shell} como interface para o kernel
\item estrutura de diretórios
\item usuário {\em root}
\item proprietários e permissões
\item principais comandos: {\tt man, ls, cp, mv, less, cat}
\item como descobrir a sintaxe dos comandos usando o {\tt man}
\end{itemize}

\paragraph{Init} Explicar como o kernel é inicializado usando os
comandos:

\begin{verbatim}
    $ ls -lh /boot
    $ uname -a
\end{verbatim}

Na linha 1 comentar sobre o {\tt grub} e o kernel que é inicializado,
com a verificação na linha 2.

\paragraph{PCB} A Listagem a seguir mostra os comandos que podem ser
utilizados para acessar a estrutura do bloco de controle de processo (PCB) do
Linux:

\begin{verbatim}
   $ mkdir linux && cd linux
   $ wget https://mirrors.edge.kernel.org/pub/linux/kernel/v4.x/linux-4.4.160.tar.gz
   $ tar xfvz linux-4.4.160.tar.gz
   $ less -p "struct task_struct \{" include/sched.h
\end{verbatim}

Na linha 4, editamos o arquivo {\tt include/sched.h} e procuramos {\tt struct
  task\_struct}, que atua como bloco de controle do processo.

\paragraph{top} Mostrar os processos em execução utilizado o comando:

\begin{verbatim}
    $ top
\end{verbatim}


Na execução do comando {\tt top} selecionar o usuário digitando {\tt
  u}. Para terminar um processo em execução, basta digitar {\tt k} e
em seguida fornecer o identificador do processo. Verificar que os processos
podem estar nos estados {\tt executando}, {\tt dormindo}, {\tt parado},
 {\tt zumbi}.

No comando {\tt top} verificar também:

\begin{itemize}
\item utilização de recursos
\item uso da memória
\item estados dos processos
\end{itemize}

\paragraph{ps} Ver os {\em snapshots} de processos usando os comandos:

\begin{verbatim}
    $ ps aux
    $ pstree
\end{verbatim}


Na linha 1, comentar sobre os usuários que estão rodando processo, e
na linha 2 mostrar a árvore de processos.

\paragraph{{\tt fork()}} A Listagem~\ref{lst:fork} mostra o código utilizado
para criar o processo utilizando a {\em chamada de sistema\/}\ {\tt fork}. A
semântica da função {\tt fork()} deve seguir a especificação
\href{http://pubs.opengroup.org/onlinepubs/009695399/functions/fork.html}{POSIX}.

\lstinputlisting[label=lst:fork,caption={Criação de um processo usando a função {\tt fork()}.}]{../proc/fork.c} 

% \paragraph{{\tt fork()} II} A função {\tt fork()} cria um novo processo, sendo que
% todo processo deve ter um processo pai, que é a cópia ``virtual'' do processo
% pai, com relação ao código, área de dados estáticos, {\em stack} e {\em heap}. O
% processo filho irá se diferenciar do processo pai somente nas páginas de memória
% que forem alteradas. Este mecanismo é conhecido como {\em copy-on-write}
% (cópia-durante-escrita), pois somente as páginas em que houve escrita no
% processo filho serão duplicadas, possuindo sua própria cópia. A sequência de
% execução entre o processo pai e filho é indeterminada, dependendo do
% escalonamento. O programa da Listagem~\ref{lst:seq} demonstra estas
% características.

%\clstsettings
%\lstinputlisting[label=lst:seq,caption={Sequência de execução dos processo pai e filho.}]{../src/proc-sequencia-fork.c} 
Enquanto o processo estiver rodando algumas informações do PCB podem ser vistas
utilizando o comando:

\begin{verbatim}
    $ cat /proc/PID/status
\end{verbatim}


Por exemplo, excutando o comando a seguir para um processo cujo {\tt PID} é {\tt
  19599}, obtemos a saída com informações sobre o processo. As palavras na
segunda coluna foram inseridos para explicar o significado de cada linha.\\

\begin{verbatim}
    $ cat /proc/19599/status
\end{verbatim}

\begin{tabular}{ll}
\tt Name:   smtpd & \footnotesize nome do programa\\
\tt State:  S (sleeping) & \footnotesize estado\\
\tt Tgid:   19599& \footnotesize ID do grupo da \thread\\
\tt Pid:    19599& \footnotesize \thread{} ID\\
\tt PPid:   4121& \footnotesize PID do processo pai\\
%\tt TracerPid:      0& \footnotesize \\
\tt Uid:    108     108     108     108& \footnotesize ID do usuário\\
\tt Gid:    124     124     124     124&\footnotesize ID do grupo \\
\tt FDSize: 128& \footnotesize número de descritores de arquivos alocados\\
\tt Groups: 124 & \footnotesize IDs de grupos suplementares\\
\tt VmPeak:    44596 kB&\footnotesize pico do tamanho da memória virtual\\
\tt VmSize:    44592 kB& \footnotesize memória virtual usada atualmente\\
\tt VmLck:         0 kB& \footnotesize memória bloqueada\\
%\tt VmPin:         0 kB& \footnotesize\\
%\tt VmHWM:      4088 kB& \footnotesize\\
%\tt VmRSS:      4088 kB& \footnotesize\\
\tt VmData:      588 kB&\footnotesize tamanho do segmento de dados\\
\tt VmStk:       136 kB& \footnotesize tamanho da {\em stack}\\
\tt VmExe:       208 kB& \footnotesize tamanho do texto (código executável)\\
\tt VmLib:      6436 kB& \footnotesize tamanho da biblioteca compartilhada\\
\tt VmPTE:       100 kB& \footnotesize Tamanho da tabela de páginas \\
%\tt VmSwap:        0 kB& \footnotesize\\
\tt\color{gray} Threads:        1&\color{gray} \footnotesize número de \thread{}s\\
\tt\color{gray} SigQ:   0/23781& \color{gray}\footnotesize atual/máx. tamanho da fila de sinais\\
\tt\color{gray} SigPnd: 0000000000000000&\color{gray} \footnotesize Sinais pendentes por \thread\\
\tt\color{gray} ShdPnd: 0000000000000000&\color{gray} \footnotesize Sinais pendentes por processo\\
\tt\color{gray} SigBlk: 0000000000000000&\color{gray} \footnotesize Sinais bloqueados\\
\tt\color{gray} SigIgn: 0000000001001000&\color{gray} \footnotesize Sinais ignorados\\
\tt\color{gray} SigCgt: 0000000180002000&\color{gray} \footnotesize Sinais capturados\\
%\tt CapInh: 0000000000000000& \footnotesize \\
%\tt CapPrm: 0000000000000000& \footnotesize\\
%\tt CapEff: 0000000000000000& \footnotesize\\
%\tt CapBnd: ffffffffffffffff& \footnotesize\\
\tt Cpus\_allowed:   f& \footnotesize número de CPUs permitidas\\
\tt Cpus\_allowed\_list:      0-3&\footnotesize número de CPUs permitidas, formato lista\\
\tt Mems\_allowed:   00000000,00000001& \footnotesize nós de memória permitidos\\
\tt Mems\_allowed\_list:      0&\footnotesize lista de nós de memória permitidos\\
\tt voluntary\_ctxt\_switches:        5& \footnotesize trocas de contexto voluntárias\\
\tt nonvoluntary\_ctxt\_switches:     5& \footnotesize trocas de contextos
forçadas (escalonador)\\
\end{tabular}

