\date{10/5/2016}
\maketitle

\section*{Gerenciamento de Entrada e Saída (E/S)}

\paragraph{Interrupções.} Verifique os números de requisições de  interrupções
pelo comando:

\CMD{cat /proc/interrupts}

\paragraph{escalonamento de E/S}~Verifique qual a política de
escalonamento de E/S usando o comando:

\CMD{cat /sys/block/sda/queue/scheduler}

\paragraph{sync.} Use o comando {\tt sync} para forçar
a atualização dos blocos modificados na memória secundária. Veja seu
manual pelo comando 

\CMD{man sync}

\paragraph{iostat.} Mostra a estatística de E/S. 

\begin{itemize}
\item Para ler informações estendidas a cada 5 segundos

  \CMD{iostat -x 5}

\item Para ler informações apenas sobre o disco a cada 5 segundos

  \CMD{iostat -dx 5}

\item Informações em MB

  \CMD{iostat -m}

\end{itemize}

\paragraph{hdparm.} Lê ou ajusta parâmetros de dispositivos SATA/IDE.

\begin{itemize}
\item Para ler as informações do disco

  \CMD{sudo hdparm -I /dev/sda | less} 

  \noindent a opção {\tt -I} deve ser usada, {\tt less} é usado para
  paginação da saída.

\item Para testar a velocidade, deve ser usada a opção {\tt -t}:

  \CMD{sudo hdparm -t /dev/sda} 

\item O comando a seguir verifica se o disco está usando um {\em buffer} 
  para transferência dos dados

  \CMD{sudo hdparm -W /dev/sda}

  \noindent  aumentando a velocidade de transmissão  se estiver utilizando.

\item Os discos modernos conseguem ler alguns setores a mais, que serão os mais 
  prováveis a serem requisitados. A opção {\tt -a} controla a quantidade 
  de setores a mais a serem lidos. Pelo comando

  \CMD{hdparm -a256 /dev/sda}

  serão lidos 256 setores a mais.

\end{itemize}

