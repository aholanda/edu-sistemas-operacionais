\documentclass{beamer}

\usepackage[utf8]{inputenc}
\usepackage[brazil]{babel}
\usepackage{tikz}
\usetikzlibrary{shapes}

\author{Adriano J. Holanda}
\title{E/S: Dispositivos}
\date{13 de novembro de 2012}

\begin{document}

\frame{\maketitle}

\section{Introdução}

\begin{frame}{Entrada e Saída: E/S}  
  \usetikzlibrary{shapes}%add to preambule if the main file does not compile

\begin{tikzpicture}[control/.style={text width=1.65cm,align=center,font=\footnotesize,draw},
    every path/.style={draw},device/.style={fill=yellow,font=\scriptsize,minimum width=1cm},
    device connection/.style={dashed},
    disk/.style={black!80,font=\scriptsize,cylinder,minimum width=1.5cm,rotate=90,draw},
    os module/.style={minimum size=3.25cm,circle,red,line width=4,text width=2cm,draw},
    module label/.style={red,font=\bf\scriptsize}]

    \def\dx{3cm}
    \def\dy{2cm}
    
    \node[control] (proc) at (0,0) {processador};
    \node[control] (disk controller) at (\dx,0) {controladora de disco};
    \node[control] (usb controller) at (2*\dx,0) {controladora USB};
    \node[control] (video card) at (3*\dx,0) {placa de vídeo};
    \node[control,fill=green!40!black] (mem) at (1.5*\dx,-\dy) {memória};

    \path (proc) -- +(0,-0.5*\dy) -- +(3*\dx,-0.5*\dy) -- +(video card);
    \path (\dx,-0.5*\dy) -- +(disk controller);
    \path (2*\dx,-0.5*\dy) -- +(usb controller);
    \path (1.5*\dx,-0.5*\dy) -- +(mem);

    \node[disk] (hd) at (\dx,\dy) {disco rígido};
    \path[device connection] (disk controller) -- (hd);

    \node[device] (mouse) [above of=usb controller,yshift=\dy] {mouse};
    \node[device,fill=gray] (printer) [right of=mouse,xshift=.15*\dx] {impressora};
    \node[device,fill=yellow!70] (keyboard) [left of=mouse,xshift=-.15*\dx] {teclado};
    \path[device connection] (usb controller) -- (mouse);
    \path[device connection] (usb controller) -- (printer);
    \path[device connection] (usb controller) -- (keyboard);

    \node[device,fill=blue!30] (monitor) [above of=video card,yshift=.5*\dy] {\bf monitor};
    \path[device connection] (video card) -- (monitor);

    \node<2>[os module] (proc module) at (proc) {};
    \node<2>[module label] [above of=proc module,yshift=.5*\dy] {gerenciamento de processos};

    \node<3>[os module] (mem module) at (mem) {};
    \node<3>[module label] [below of=mem module,yshift=-.5*\dy] {gerenciamento de memória};

    \node<4>[module label,minimum width=7.5cm,dashed,draw] [above of=usb controller] {gerenciamento de entrada e saída (E/S)};
  \end{tikzpicture}

\end{frame}

\begin{frame}{Subsistema de entrada e saída}

{\small Fonte:~\cite{ufrgs2008}}
\begin{center}
  \begin{tikzpicture}[layer/.style={font=\small,minimum height=1cm,minimum width=8cm,draw},
    driver/.style={minimum height=1cm, align=center,font=\footnotesize,text width=1.25cm,draw},
    coda/.style={<->,>=latex}]
    
    \node[layer] (userlevel)  {E/S: nível de usuário};    
    \draw[dashed] (userlevel.west)+(-.25,-.65) -- +(8cm,-.65);
    \draw[coda] (userlevel.west)+(-.25,-.65) -- +(-.25,-4.45cm);
    
    \draw[coda] (userlevel.east)+(.25,.5) -- +(.25,-4.45cm);

    \node[layer] (io0) [below of=userlevel,yshift=-.25cm] {E/S independente do dispositivo};
    \node[layer,fill=yellow] (api) [below of=io0,yshift=-.25cm] {Interface padrão para {\em drivers} de dispositivo (API)};
    
    \foreach \x/\l in {-2/driver disco,-1/driver USB,0/driver rede,1/$\ldots$,2/driver teclado} {
      \node[driver] (driver\x) [below of=api, xshift=1.6*\x cm, yshift=-.25cm] {\l};
    }
    \node[white, fill=black, layer] (hardware) [below of=driver0, yshift=-.25cm] {Hardware};

    \node[rotate=90] [left of=driver-2,yshift=1.5cm,xshift=2.25cm] {Sistema operacional};
    \node [rotate=-90] [right of=io0,yshift=4.5cm,xshift=-1cm] {Software};

  \end{tikzpicture}
\end{center}

\end{frame}

\begin{frame}{Classificação das camadas de E/S}
  
  \begin{block}<1->{Dispositivos de caracter}
    São dispositivos cujo fluxo de dados ocorre de forma sequencial,
    um byte após o outro.\\
    \smallskip
    Exemplo: teclado, porta serial.
  \end{block}


  \begin{block}<2>{Dispositivos de bloco}
    Os dados são acessados de forma aleatória em pedaços de 
    tamanho fixo chamado blocos.\\
    \smallskip
    Exemplo: Disco rígido, memória flash, disquete, leitor de Blu-ray.    

  \end{block}

\end{frame}

\section{Dispositivos de bloco}

\begin{frame}{Dispositivos de bloco}
  {Conceitos essenciais}
  
  \begin{itemize}
    \item A menor unidade endereçável em um dispositivo de bloco é um \alert{setor}.
    \item O tamanho mais comum do setor é \alert{512} bytes. Porém, alguns dispositivos
      possuem tamanhos diferentes, por exemplo, muitos CD-ROMs possuem setores de 2-KB.
    \item Os blocos são armazenados temporariamente em um zona de memória chamada 
      \alert{\em buffer}, para ajuste entre as diferenças na velocidade de transferência entre
      as camadas dos subsistemas. \\
      \smallskip
      {\small Por exemplo, se a placa de rede suporta envio de blocos 
      de 4-KB, e o SO necessita enviar 64-KB, estes são armazenados no {\em buffer} até
      o término da transmissão.}

    \end{itemize}
        
\end{frame}

\frame{\author{}\title{Dispositivos de Bloco: \\Disco Rígido}\date{}\maketitle}

\begin{frame}{Busca em disco rígido}
\begin{center}
\input{../../cod/io/img/platter}
\end{center}
\end{frame}

\begin{frame}{Escalonamento das requisições de E/S}
  
  \begin{itemize}
  \item O \alert{escalonador de E/S} gerencia a fila de requisições
    dos dispositivos de bloco, decidindo qual requisição é despachada,
    com o objetivo de maximizar a performance global de E/S do
    sistema.
  \item O escalonadores de E/S realizam 2 ações principais sobre as
    requisições para minimizar as buscas:
    \begin{itemize}
    \item \alert{Fusão}: agrupamento de 2 ou mais requisições ao mesmo bloco;
    \item \alert{Ordenação}: classifica as requisições de acordo com
      algum critério.
    \end{itemize}
  \end{itemize}
\end{frame}

\begin{frame}{Algoritmos de escalonamento de E/S}{Disco rígido}
  \footnotesize
  \begin{enumerate}[<+-| alert@+>]
  \item \alert{FCFS}({\em firt-come first-served}): as requisições são
    atendidas conforme a ordem de chegada na fila;
  \item \alert{SSTF} ({\em shortest seek time first}): as requisições
    são atendidas de acordo com o menor tempo de acesso, e são
    reordenadas constantemente, para levar em conta a posição atual do
    cabeçote, privilegiando os setores mais próximos à posição
    corrente na reordenação da fila de requisições.
    {\color{gray}\small Desvantagem: Suscetível à postergação
      indefinida ({\em starvation}).}
  \item \alert{SCAN}: é uma variação do SSTF que se diferencia por
    adotar um sentido de varredura preferencial, como por exemplo do
    mais interno para o mais externo. Ao atingir o mais interno,
    inverte-se o sentido e novas requisições no sentido contrário são
    atendidas na próxima varredura. Também é conhecido como algoritmo
    do \alert{elevador}.
  \item \alert{C-SCAN}: variação do SCAN que se diferencia por adotar
    o sistema de varredura em somente uma direção. Por exemplo, se o
    cabeçote atingir o cilindro mais interno, é então reposicionado
    no cilindro mais externo e a varredura é realizada novamente.
  \end{enumerate}

\end{frame}

\begin{frame}{Exercício}

  \begin{enumerate}
  \item Dada as seguintes sequências de requisições para um 
    disco com $100$ trilhas:
    \begin{center}
      $44, 20, 95, 4, 50, 52, 47, 61, 87, 25$,
    \end{center}
    onde a posição atual dos cabeçotes é $50$, calcule a
    quantidade de movimentos dos cabeçotes para atender às
    requisições utilizando os seguintes algoritmos:
    \begin{enumerate}
    \item FCFS;
    \item SSTF;
    \item SCAN;
    \item C-SCAN.
    \end{enumerate}
  \end{enumerate}
  
\end{frame}

\begin{frame}
  
  \begin{thebibliography}{5}
  \bibitem[Oliveira, 2008]{ufrgs2008}
    {\em Sistemas Operacionais}.
    \newblock Rômulo Silva de Oliveira, Alexandre da Silva Carissimi e 
    Simão Sirineo Roscani.
    \newblock Editora Bookman, 2008.
  
  \bibitem[LKD3]{lkd3} 
    {\em Linux Kernel Development}.
    \newblock Robert Love.
    \newblock Addison Wesley Publisher, 2010.

    \bibitem[Garrido, 2008]{garrido2008}
      {\em Principles of Modern Operating Systems}.
      \newblock José M.\ Garrido, Richard Schlesinger.
      \newblock Jones and Bartlett Publishers, 2008.

  \end{thebibliography}

\end{frame}


\end{document}
