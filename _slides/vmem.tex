
%%%%%%%%%%%%%%%%%%%% VIRTUAL MEMORY %%%%%%%%%%%%%%%%%%%%

\lecture{Memória Virtual}{mem:virtual}

\frame{\title{\insertlecture}\maketitle}

\transition{Paginação}{mem:paging}

%[rlove] pg 231
\begin{frame}{Páginas}

  O SO lida com \alert{páginas} de memória física ao invés de bytes
  ou palavras. As páginas são gerenciadas pela MMU ({\small {\em
      Memory Management Unit} -- Unidade de Gerenciamento de
    Memória}). O tamanho das páginas variam de acordo com a
  arquitetura do processador, mas para a maioria das arquiteturas
  temos:

\begin{itemize}
\item 32 bits: páginas de 4KB;
\item 64 bits: páginas de 8KB.
\end{itemize}

\end{frame}

\begin{frame}{Paginação}
\centering
\def\W{1cm}
\def\H{0.25cm}
\def\w{.25cm}
\def\h{\w}
\usetikzlibrary{matrix,shapes}
\begin{tikzpicture}[every node/.style={font=\scriptsize},
  page/.style={minimum width=\W,minimum height=\H,draw},
  00/.style={fill=yellow}, 
  01/.style={fill=green}, 
  10/.style={fill=cyan},
  11/.style={fill=black},
  memcel/.style={minimum width=\w,minimum height=\h,text=white,font=\bf,draw},
  bitlabel/.style={font=\scriptsize, align=center,minimum height=2*\H},
  ptheader/.style={bitlabel, rotate=90, font=\tiny, draw}
  ]
  \scriptsize
  \newcounter{vmc}\setcounter{vmc}{0}
  \foreach \x/\addr in {0/11,1/10,2/01,3/00} {
    \node[page,\addr] (vm\x) at (0,\x*\H) {};
    \node[font=\tt] at (-\W/1.5,\x*\H) {\addr};
    \addtocounter{vmc}{1};
  }
  \node[font=\footnotesize,align=center, text width=\W] [above of=vm3,yshift=-\h] {memória virtual};
      
  \matrix at (\W/8,-10*\H) (pagetable) [matrix of nodes,column sep=\w,minimum width=2*\w,fill=yellow!50] {
    \node[ptheader] {bit de validade}; & \node[ptheader] {bit de sujeira}; & 
    \node[ptheader] {endereço lógico}; & \node[ptheader] {endereço físico};\\
    1 & 0 & \tt 00 & (3,2)\\
    1 & 0 & \tt 01 & (2,6)\\
    0 & 0 & \tt 10 & (5,3)\\
    1 & \node (ptbase) {1}; & \tt 11 & (1,0)\\
  };
  \node [below of=ptbase] {Tabela de páginas};
  

  \def\deltax{4.5}
  \def\PHYSMX{\deltax*\W+\x*\w}
  \foreach \x in {0,...,7} {
    \node at (\PHYSMX,6*\H) {\x};
    \node at (.925*\deltax*\W,\h*-\x+5.*\H) {\x};
    \foreach \y in {0,...,7} {
      \node[memcel] (physmem\x\y) at (\PHYSMX,\y*\h-2*\H) {};
    }
  }
  \node[font=\footnotesize] [above of=physmem37] {DRAM};

  
  \node[memcel,00] at (physmem35) {};
  \node[memcel,01] at (physmem21) {};
  %\node[memcel,10] at (physmem54) {}; % stored in the the disk
  \node[memcel,11,font=\tiny] at (physmem17) {8};

  \node[shape border rotate=90,minimum height=6*\H,minimum width=2*\W,cylinder,draw] (disk)
  [below of=physmem30, xshift=\W/6, yshift=-2.5*\H]  {};
  \node[font=\footnotesize] [below of=disk,yshift=-1.5*\H] {disco magnético};
  
  \foreach \x/\addr in {-6/00,-5/01,-4/10,-3/11} {
    \node[memcel,\addr,font=\tiny] [right of=disk,xshift=\x*\w] {\ifnum\addr=11 -1\fi};
  }
\end{tikzpicture}
% Local Variables:
% main: ../notes
% End:

\end{frame}


%\begin{frame}{Unidade de Gerenciamento de Memória \only<2>{com Buffer/Cache}}
%\begin{center}

%\begin{tikzpicture}
%  \tikzset{every node/.style={draw,minimum height=0.5cm, minimum
%      width=1.5cm,font={\textbf\footnotesize}}, every
%    path/.style={->,draw,>=latex}, lebal/.style={text width=2.5cm,
%      align=center, font=\scriptsize,draw=white}}

%  \node[fill=gray!50,draw] (proc) {processador}; 
%  \node[] (tlb) [right of=proc,xshift=2cm,color=white,fill=red,draw=red] {TLB};
%  \node[minimum height=1cm,color=white,fill=green!40!black,draw=green!40!black]
%  (mmu) [below of=tlb,yshift=-1.25cm] {MMU};

%  \path[<->] (proc) -> (tlb);
%  \path[->,color=green!50!black] (tlb.south east) -> (mmu.north east);
%  \path[->,color=red!60!black] (mmu.north west) -> (tlb.south west);

%  \foreach \y in {0,1,2,3,4,5} {
%    \node[fill=yellow!20] (page\y) [right
%    of=mmu,yshift=-5mm*\y,xshift=4cm] 
%    {\scriptsize p\'agina \y};
%  }
%  \node[lebal] [below of=page5,yshift=.25cm] {Tabela de páginas};

%  \path[<->,color=blue!50!black] (mmu.east) -> (page1.west);

%  \node[lebal] (mmulabel) [below of=mmu] 
%  {\em \scriptsize MMU -- memory management unit};

%  \node[lebal] [above of=tlb,yshift=-.25cm] 
%  {\em \scriptsize TLB -- Translation Look-aside buffer};
%\end{tikzpicture}

%\end{center}

%\end{frame}

\section*{Algoritmos de substituição de páginas}

\begin{frame}{Algoritmos de substituição de páginas}
\begin{description}
\item[FIFO ({\em First In First Out}):] a página a ser substituída
  durante a paginação será sempre a mais antiga;
\pause\item[Segunda chance:] parecido com o FIFO, porém, se o bit de
  referência da primeira página for $1$, este é zerado e a página
  colocada no fim da fila. Se for $0$, a página é imediatamente
  substituída.
\pause\item[Relógio:] parecido com o algoritmo de segunda chance, porém, as
  páginas são mantidas em uma lista circular, com um ponteiro
  percorrendo a lista. Se houver necessidade de substituição, a página
  apontada é analisada, se o bit de referência for $0$, esta é
  escolhida, senão, o bit de referência é zerado e ponteiro continua
  atá a próxima página em que o bit seja $0$.
\pause\item [LRU ({\em Last Recently Used}):] a página a ser substituída é a
  última utilizada recentemente.
\end{description}
\end{frame}

% Fonte: https://www.cs.uic.edu/~jbell/CourseNotes/OperatingSystems/9_VirtualMemory.html
% 9.6 Thrashing

% If a process cannot maintain its minimum required number of frames,
% then it must be swapped out, freeing up frames for other
% processes. This is an intermediate level of CPU scheduling.  But what
% about a process that can keep its minimum, but cannot keep all of the
% frames that it is currently using on a regular basis? In this case it
% is forced to page out pages that it will need again in the very near
% future, leading to large numbers of page faults.  A process that is
% spending more time paging than executing is said to be thrashing.
% 9.6.1 Cause of Thrashing

% Early process scheduling schemes would control the level of
% multiprogramming allowed based on CPU utilization, adding in more
% processes when CPU utilization was low.  The problem is that when
% memory filled up and processes started spending lots of time waiting
% for their pages to page in, then CPU utilization would lower, causing
% the schedule to add in even more processes and exacerbating the
% problem! Eventually the system would essentially grind to a halt.
% Local page replacement policies can prevent one thrashing process from
% taking pages away from other processes, but it still tends to clog up
% the I/O queue, thereby slowing down any other process that needs to do
% even a little bit of paging ( or any other I/O for that matter. )

% Figure 9.18 - Thrashing

% To prevent thrashing we must provide processes with as many frames as
% they really need "right now", but how do we know what that is?  The
% locality model notes that processes typically access memory references
% in a given locality, making lots of references to the same general
% area of memory before moving periodically to a new locality, as shown
% in Figure 9.19 below. If we could just keep as many frames as are
% involved in the current locality, then page faulting would occur
% primarily on switches from one locality to another. ( E.g. when one
% function exits and another is called. )


\begin{frame}{\em Trashing}

  Quando um processo não possui o número de quadros ({\em frames}) de
  memória necessários a cada escalonamento para sua execução, o
  processo passa a maior parte do tempo esperando pelas substituição
  de páginas do que em execução. Este efeito é chamado de {\em
    trashing}.

\end{frame}

