
\transition{Sistema de arquivos virtual}{fs:virtual}

\title{\insertlecture}

\begin{frame}{\insertlecture}{\only<1>{Definição}\only<2>{Entidades}}
\small
{\only<2>{\color{gray}}
  É o sistema utilizado em sistemas baseados em {\sc Unix} para
  padronizar as abstrações e operações dos diferentes sistemas de
  arquivos com as chamadas de sistemas.
}\bigskip

\only<2>{
  O sistema de arquivos virtual possue quatro entidades básicas:
  
  \begin{description}
 \item[Superbloco]~estrutura que contém atributos do sistema de
   arquivos e informações sobre os arquivos.
 \item[Inode:]~estrutura de dados separada do arquivo que contém
   atributos a respeito dos arquivos;
 \item[Diretórios:]~recipiente de arquivos correlacionados ou outros
   diretórios chamados subdiretórios;
 \item[Arquivos:]~sequência ordenada de bytes;
\end{description}
}

\end{frame}

\begin{frame}{\insertlecture}{Portabilidade}

Os sistemas de arquivos possuem interface de programação comum para
melhorar a portabilidade do sistema. As principais operações são:

\begin{itemize}
\item {\tt open()}\item {\tt read()}\item {\tt write()}
\item {\tt create()}\item {\tt delete()}
\end{itemize}

O funcionamento de cada função depende da entidade à qual a interface
está sendo aplicada.

\end{frame}


\begin{frame}{\insertlecture}\small
O sistema de arquivos virtual implementa a interface entre programas
do espaço do usuário e o sistema de arquivos.

% A Figura~\ref{fig:2io:vfs} mostra a chamada de sistema {\tt write()} que é
% atendida pelo subsistema de arquivo virtual e utiliza a função
% correspondente para o sistema de arquivos em que o meio físico está
% formatado para gravação dos dados.

\only<1>{
%%% Local Variables: 
%%% mode: latex
%%% TeX-master: "../notes"
%%% End: 


\begin{tikzpicture}
  \tikzset{every node/.style={text centered},
    every edge/.style={->,draw,>=latex}, thelabel/.style={font=\footnotesize}} 
  \footnotesize 
  \node[draw] (w) {{\tt write()}};
  \node[text width=2cm,draw] (vfs) [right of=w,xshift=20mm] {Sistema de
    arquivos virtual};
  \node[shape=circle,minimum size=12mm,draw] (cd) [right of=vfs,
  xshift=25mm] {};
  \node[shape=circle,minimum size=4mm,draw] at (cd) {};
  \node[thelabel] [above of=cd] {CD/DVD};
  \node[shape=cylinder,shape border rotate=90,minimum width=1cm,
  minimum height=1.75cm,draw] (disk) [above of=cd, yshift=15mm]
  {};
  \node[thelabel] [above of=disk,yshift=5mm] {Disco};
  \node[shape=rectangle,minimum width=.75cm,
  minimum height=1.25cm,draw] (pendrive) [below of=cd, yshift=-15mm] {};
  \node[thelabel] [above of=pendrive] {Pendrive};

  %% setas
  \path (w) edge (vfs);
  \path (vfs) edge node[above] {\scriptsize ISO-9660} (cd);
  \path (vfs) edge node[above, rotate=25] {ext3} (disk);
  \path (vfs) edge node[above,rotate=-25] {\scriptsize FAT32} (pendrive);

\end{tikzpicture}}

\only<2>{\bigskip O sistema Microsoft\textsuperscript{\textregistered} Windows
  não possui sistema de arquivos virtual, portanto somente sistemas de
  arquivos que possuem suporte pelo sistema tais como FAT, FAT32 e
  NTFS são acessíveis.}

\end{frame}

\begin{frame}{\insertlecture}{Interfaces com o meio físico}

Chamada de função:
\begin{tt}
  int ret = write(fd, buf, len);\bigskip
\end{tt}

% representada abstratamente na Figura~\ref{fig:2io:vfs:sys} grava {\tt len}
% bytes apontados por {\tt buffer} na posição atual do arquivo
% representado pelo descritor de arquivos.

% Note que a os dados são repassado para o sistema de arquivos virtual
% que se encarrega de distribuir para a função adequada ao sistema de
% arquivo no qual se deseja gravar.


%%% Local Variables: 
%%% mode: latex
%%% TeX-master: "../notes"
%%% End: 


\begin{tikzpicture}
  \tikzset{every node/.style={text centered},
    every edge/.style={->,draw,>=latex}, thelabel/.style={font=\footnotesize}} 
  \small \node[draw] (w) {{\tt write()}};
  \node [text width=1cm,below of=w,yshift=-5mm] {\scriptsize Espaço do usuário};
  \node[draw]  (wsys) [right of=w,xshift=15mm] {\tt sys\_write};
  \node [text width=1.5cm,below of=wsys,yshift=-5mm] {\scriptsize Sistema de arquivos virtual};
  \node[text width=2cm,draw]  (fs) [right of=wsys,xshift=20mm] 
  {Função do sistema de arquivos};
  \node [text width=1.5cm,below of=fs,yshift=-5mm] {\scriptsize Sistema de arquivos};
  \node[shape=cylinder,shape border rotate=90,text width=1cm,
  aspect=.5,draw] (disk) [right
  of=fs,xshift=20mm] {Meio físico};

  %% setas
  \path (w) edge (wsys);
  \path (wsys) edge (fs);
  \path (fs) edge (disk);
  

\end{tikzpicture}

\end{frame}
