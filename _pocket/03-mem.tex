
\chapter{Gerenciamento de memória}

\section{Paginação}

Paginação é a técnica de dividir a memória utilizada pelo SO (Sistema
Operacional) em blocos, mapeando o endereço físico a um endereço
lógico ou virtual, permitindo utilização não-contígua dos {\em frames}
de memória e não carregamento de todos os blocos da memória secundária
para a memória principal.

A página é a menor unidade de memória que o SO gerencia, embora, o
processador manipule bytes, o SO lida com blocos que possuem os
seguintes tamanhos de acordo com a arquitetura, padronizados para
garantir portablidade:

- 32-bit: 4KB;
- 64-bit: 8KB.

Se a memória principal possuir tamanho de 8GB em uma arquitetura de
64-bit, haverá $1.048.576$ páginas distintas.

O processo de carregamento das páginas é intermediado pela {\bf
Unidade de Gerenciamento de Memória} ({\em MMU -- Memory Management
Unit}) que consulta uma {\bf tabela de páginas} para verificar 
a localização física do endereço lógico (virtual) requisitado.

\begin{figure}
\def\W{1cm}
\def\H{0.25cm}
\def\w{.25cm}
\def\h{\w}
\usetikzlibrary{matrix,shapes}
\begin{tikzpicture}[every node/.style={font=\scriptsize},
  page/.style={minimum width=\W,minimum height=\H,draw},
  00/.style={fill=yellow}, 
  01/.style={fill=green}, 
  10/.style={fill=cyan},
  11/.style={fill=black},
  memcel/.style={minimum width=\w,minimum height=\h,text=white,font=\bf,draw},
  bitlabel/.style={font=\scriptsize, align=center,minimum height=2*\H},
  ptheader/.style={bitlabel, rotate=90, font=\tiny, draw}
  ]
  \scriptsize
  \newcounter{vmc}\setcounter{vmc}{0}
  \foreach \x/\addr in {0/11,1/10,2/01,3/00} {
    \node[page,\addr] (vm\x) at (0,\x*\H) {};
    \node[font=\tt] at (-\W/1.5,\x*\H) {\addr};
    \addtocounter{vmc}{1};
  }
  \node[font=\footnotesize,align=center, text width=\W] [above of=vm3,yshift=-\h] {memória virtual};
      
  \matrix at (\W/8,-10*\H) (pagetable) [matrix of nodes,column sep=\w,minimum width=2*\w,fill=yellow!50] {
    \node[ptheader] {bit de validade}; & \node[ptheader] {bit de sujeira}; & 
    \node[ptheader] {endereço lógico}; & \node[ptheader] {endereço físico};\\
    1 & 0 & \tt 00 & (3,2)\\
    1 & 0 & \tt 01 & (2,6)\\
    0 & 0 & \tt 10 & (5,3)\\
    1 & \node (ptbase) {1}; & \tt 11 & (1,0)\\
  };
  \node [below of=ptbase] {Tabela de páginas};
  

  \def\deltax{4.5}
  \def\PHYSMX{\deltax*\W+\x*\w}
  \foreach \x in {0,...,7} {
    \node at (\PHYSMX,6*\H) {\x};
    \node at (.925*\deltax*\W,\h*-\x+5.*\H) {\x};
    \foreach \y in {0,...,7} {
      \node[memcel] (physmem\x\y) at (\PHYSMX,\y*\h-2*\H) {};
    }
  }
  \node[font=\footnotesize] [above of=physmem37] {DRAM};

  
  \node[memcel,00] at (physmem35) {};
  \node[memcel,01] at (physmem21) {};
  %\node[memcel,10] at (physmem54) {}; % stored in the the disk
  \node[memcel,11,font=\tiny] at (physmem17) {8};

  \node[shape border rotate=90,minimum height=6*\H,minimum width=2*\W,cylinder,draw] (disk)
  [below of=physmem30, xshift=\W/6, yshift=-2.5*\H]  {};
  \node[font=\footnotesize] [below of=disk,yshift=-1.5*\H] {disco magnético};
  
  \foreach \x/\addr in {-6/00,-5/01,-4/10,-3/11} {
    \node[memcel,\addr,font=\tiny] [right of=disk,xshift=\x*\w] {\ifnum\addr=11 -1\fi};
  }
\end{tikzpicture}
% Local Variables:
% main: ../notes
% End:

\label{fig:paging}
\caption{Paginação.}
\end{figure}

A Figura~\ref{fig:paging} mostra as informações armazenadas
na tabela de páginas necessárias para localização e manutenção
da integridade dos dados armazenados nas memórias primária
(DRAM--{\em Dynamic Random Access Memory}) e secundária 
(disco magnético).

Quando a página contida no endereço lógico {\tt 01} for requisitada
junto à MMU, o conteúdo do endereço físico da memória principal
$(2,6)$ é retornado ao requisitante.

O campo do {\bf bit de validade} armazena a informação sobre as
páginas carregadas na memória principal. A página {\tt 10}, por
exemplo, não se encontra na memória principal, portanto, se esta for
requisitada, terá que ser carregado da memória secundária para a
principal e o valor do bit atribuído para $1$.

O {\bf bit de sujeira (dirty)} indica se a página foi modificada na
memória principal, e esta modificação ainda não foi escrita na memória
secundária. Por exemplo, se a página armazenasse somente valores
inteiros, a página {\tt 11} foi carregado da memória secundária para a
principal com o valor $-1$, após alguns instantes, seu valor é
modificado para $8$.  O bit de sujeira é modificado para o valor $1$
indicando que o estado desta página na memória principal é diferente
da existente na secundária. Há a necessidade de atualizar o estado da
página {\tt 11} na memória secundária.

O bit de sujeira assume um papel importante, também se a página não
for modificada, pois, indica que a página não precisa ser atualizada
na memória secundária, economizando uma operação de escrita de E/S
(Entrada/Saída).

{\color{blue}\bf Exercício.} Dois processos possuem as seguintes páginas lógicas:

\begin{center}
$P_1= \{a,b,c,d\}$
$P_2= \{1,2,3\}$
\end{center}

Supondo que o sistema operacional dispõe de 4 páginas físicas
para alocar as páginas lógicas, faça um esboço gráfico
para a ocupação das páginas lógicas que forem requisitadas
na seguinte sequência de operações do processador:

\begin{center}
{\tt\{R(2),W(3),R(1),R(a),R(b),W(3),W(1),W(1),R(c),R(d),W(1),R(a)\}}
\end{center}
utilizando os seguintes algoritmos de substituição de p\'aginas:\\
\begin{enumerate}
\item FIFO;
\item Segunda chance;
\item Relógio;
\item LRU.
\end{enumerate}

\noindent onde\\
\noindent {\tt R(p)} -- significa {\bf leitura} ({\em read}) da página {\tt p} e\\
\noindent {\tt W(p)} -- significa {\bf escrita} ({\em write}) na página {\tt p}.\\
\bigskip
Calcule a taxa de aus\^encia de p\'aginas (porcentagem de {\em page
miss/fault}) e a quantidade de operações de escrita na memória
secundária para ambos os algoritmos.


