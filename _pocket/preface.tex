

\chapter*{Prefácio}
\label{ch:preface}

A intenção deste livro é fornecer os princípios básicos pelos quais os
sistemas operacionais são guiados em seu projeto. Para atingir este
difícil objetivo, alguns detalhes mais específicos de alguns sistemas
operacionais são omitidos, e outros que constituem avanços na forma de
projetar sistemas são adicionados, mesmo que sejam específicos de
determinados sistemas, mas mostram soluções engenhosas para problemas
comuns em sistemas operacionais.

 Este livro tenta ser o mais agnóstico possível em termos de explicar
 determinados conceitos baseados apenas em um único sistema
 operacional. Porém, houve a necessidade do autor de concentrar em
 alguns momentos, a explicação focada para a implementação adotada por
 um sistema operacional. Esta decisão deve-se simplesmente ao fato do
 acesso ao código-fonte do sistema facilitar a explicação em um nível
 maior de detalhamento.

\paragraph{Especificação}~Alguns algoritmos ou processos são especificados
   usando a linguagem de especificação lógica temporal \TLAPLUS~({\it
     Temporal Logic of Actions\/}). O objetivo é formalizar alguns
   conceitos comuns em sistemas operacionais, reduzindo a ambiguidade
   e falta de critério na elucidação das relações entre os
   conceitos. Àqueles que não se adaptarem à linguagem, podem seguir
   somente a linguagem natural e as figuras sem prejuízo de
   entendimento. A linguagem de especificação fornece um modo poderoso
   de fixar o entendimento pelas restrições impostas pela lógica de
   predicados e teoria de conjuntos.

\paragraph{Programas}~Todos os programas listados foram codificados na
   liguagem de programação \CEE, e ilustram como a computação de
   determinados mecanismos poderia ser implementada, sem a pretensão
   de ser eficiente ou simular de forma fiel as estruturas de dados
   utilizadas pelos sistemas operacionais.
