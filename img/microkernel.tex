
%%% Local Variables:
%%% mode: latex
%%% TeX-master: "os-book-pt_BR"
%%% End:

\begin{figure}[ht]
\centering
  \begin{tikzpicture}[scale=.8,
    every node/.style={font=\scriptsize,anchor=south
      west,align=center,transform shape},
    server/.style={minimum size=1.5cm,circle,draw},
    every path/.style={<->,>=latex,draw}]

    \colorlet{aplicativo}{blue}
    \colorlet{servidor de}{yellow}
    \colorlet{driver de}{brown}
    \def\slabel{}

      \node<1->[color=white,fill=black,minimum width=3.5*\dx] at (0,0.85cm) (hardware) {Hardware};
      \node<2->[color=white,fill=kernel color,minimum width=3.5*\dx] at (0,1.35cm) (kernel) {kernel};

     \node<6->[minimum width=11cm,minimum height=.65cm,anchor=south,fill opacity=.4,fill=gray!30,draw=kernel color] (kernel space) 
     [above of=hardware,yshift=-.4cm] {};
     \node<6->[color=kernel color,rotate=90,text width=1.25cm] [right of=kernel,yshift=2.95*\dy,xshift=-.75cm] {\tiny espaço do kernel};

     \node<7->[minimum width=11cm,minimum height=6.5cm,anchor=south,fill opacity=.4,fill=user color,draw=user color] (user space) 
     [above of=kernel space,yshift=1.3*\dy] {};
     \node<7->[color=user color] [above of=user space,rotate=90,yshift=2.9*\dy] {espaço do usuário};

      
      \foreach \y/\ly/\slide in {1/driver de/3,2/servidor de/4,3/aplicativo/5} {
        \foreach \x in {1,2,3} {
          \ifnum\y=1
          \ifnum\x=1\def\slabel{disco}\fi
          \ifnum\x=2\def\slabel{áudio}\fi
          \ifnum\x=3\def\slabel{rede}\fi
          \fi
          \ifnum\y=2
          \ifnum\x=1\def\slabel{arquivos}\fi
          \ifnum\x=2\def\slabel{processos}\fi
          \ifnum\x=3\def\slabel{rede}\fi
          \fi
          \ifnum\x=3
          \node<\slide-> [right of=s2\y,xshift=.3cm] {\large $\dots$};
          \fi
          
          \node<\slide->[server,fill=\ly,text width=1.35cm] (s\x\y) at (2.5*\x cm,2.35*\y cm) {\ly\ \slabel};
       }
     }
 \path<3-> (kernel) -- (s11);
     \path<3-> (kernel) -- (s21);
     \path<3-> (kernel)+(3cm,0) -- (s31);
     \path<5-> (s11) -- (s12);
     \path<5-> (s12) -- (s13);
     \path<4-> (kernel) -- (s12);
     \path<4-> (kernel)+(2.25cm,0) -- (s22);
     % net
     \path<6-> (s31) -- (s32);
     \path<6-> (s32) -- (s33);
     %audio
     \path<6-> (s21) -- (s12);
     % 
     \path<5-> (s22) -- (s23);
     \path<5-> (s12) -- (s23);

     
    \end{tikzpicture}

    \label{fig:00intro:microkernel}
    \caption{Arquitetura tradicional de um kernel baseado em microkernel.}
  \end{figure}
  