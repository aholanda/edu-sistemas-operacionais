\lstset{language=C}

\lecture{Sincronização e concorrência}{sync:titlepage}

\frame{\title{\insertlecture}\titlepage}

%\frame{\frametitle{Trilha}\tableofcontents}

\lecture{Regiões críticas e condições de disputa}{sync:intro}

\section{\insertlecture}
\begin{frame}{\insertlecture}

  \begin{itemize}
    % lkd pg 162
  \item \alert{Regiões críticas} ({\em critical regions}): thechos de
    código que acessam e manipulam dados compartilhados.
  \item \alert{Condição de disputa} ({\em race condition}): quando
    duas ou mais {\em threads} estão executando simultaneamente a
    mesma região crítica, \underline{disputando} o acesso aos recursos
    compartilhados.
  \end{itemize}

\end{frame}

\begin{frame}{Exemplo}{Condição de disputa}

  \def\dist{15mm}
  \begin{tikzpicture}[memory/.style={rectangle,draw,very thick,anchor=east,font=\footnotesize},
    op/.style={rectangle,very thick,font=\scriptsize}]
    \node (mem) [memory] {7};
    \node (i) [above of=mem, yshift=-5mm,color=red!30!black!70!]
    {\large i};
    \node (t1) [right of=mem,xshift=2*\dist] {{\em Thread} $1$};
    \node (t1op1) [below of=t1,op] {lê i (7)};
    \node (t1op2) [below of=t1op1,op] {incrementa i
      (7$\rightarrow$8)};
    \node (t1op3) [below of=t1op2,op] {escreve i
      (8)};
    \node (t1op4) [below of=t1op3,op] {--};
    \node (t1op5) [below of=t1op4,op] {--};
    \node (t1op6) [below of=t1op5,op] {--};


    \node (t2) [right of=t1, xshift=\dist] {{\em Thread} $2$};
    \node (t2op1) [below of=t2,op] {--};
    \node (t2op2) [below of=t2op1,op] {--};
    \node (t2op3) [below of=t2op2,op] {--};
    \node (t2op4) [below of=t2op3,op] {lê i (8)};
    \node (t2op5) [below of=t2op4,op] {incrementa i
      (8$\rightarrow$9)};
    \node (t2op6) [below of=t2op5,op] {escreve i
      (9)};

    \path (t1) edge[->] (t1op1);
    \path (t1op1) edge[->] (t1op2);
    \path (t1op2) edge[->] (t1op3);
    \path (t1op3) edge[->] (t1op4);
    \path (t1op4) edge[->] (t1op5);
    \path (t1op5) edge[->] (t1op6);

    \path (t2) edge[->] (t2op1);
    \path (t2op1) edge[->] (t2op2);
    \path (t2op2) edge[->] (t2op3);
    \path (t2op3) edge[->] (t2op4);
    \path (t2op4) edge[->] (t2op5);
    \path (t2op5) edge[->] (t2op6);
  \end{tikzpicture}

\end{frame}

\begin{frame}{Exemplo}{Condição de disputa}
  {\color{red} \small Variável global {\tt i} ao final da disputa fica em estado inconsistente.}
  \def\dist{15mm}
  \begin{tikzpicture}[memory/.style={rectangle,draw,very thick,anchor=east,font=\footnotesize},
    op/.style={rectangle,very thick,font=\scriptsize}]
    \node (mem) [memory] {7};
    \node (i) [above of=mem, yshift=-5mm,color=red!30!black!70!]
    {\large i};
    \node (t1) [right of=mem,xshift=2*\dist] {{\em Thread} $1$};
    \node (t1op1) [below of=t1,op] {lê i (7)};
    \node (t1op2) [below of=t1op1,op] {incrementa i
      (7$\rightarrow$8)};
    \node (t1op3) [below of=t1op2,op] {--};
    \node (t1op4) [below of=t1op3,op] {escreve i
      (8)};
    \node (t1op5) [below of=t1op4,op] {--};
    \node (t1op6) [below of=t1op5,op] {--};


    \node (t2) [right of=t1, xshift=\dist] {{\em Thread} $2$};
    \node (t2op1) [below of=t2,op] {lê i (7)};
    \node (t2op2) [below of=t2op1,op] {--};
    \node (t2op3) [below of=t2op2,op] {incrementa i
      (7$\rightarrow$8)};
    \node (t2op4) [below of=t2op3,op] {--};
    \node (t2op5) [below of=t2op4,op] {--};
    \node (t2op6) [below of=t2op5,op] {escreve i
      (8)};

    \path (t1) edge[->] (t1op1);
    \path (t1op1) edge[->] (t1op2);
    \path (t1op2) edge[->] (t1op3);
    \path (t1op3) edge[->] (t1op4);
    \path (t1op4) edge[->] (t1op5);
    \path (t1op5) edge[->] (t1op6);

    \path (t2) edge[->] (t2op1);
    \path (t2op1) edge[->] (t2op2);
    \path (t2op2) edge[->] (t2op3);
    \path (t2op3) edge[->] (t2op4);
    \path (t2op4) edge[->] (t2op5);
    \path (t2op5) edge[->] (t2op6);
  \end{tikzpicture}

\end{frame}

\def\thetitle{Bloqueio}
\section{\thetitle}
\begin{frame}{\thetitle}{Operações atômicas}

  \def\dist{15mm}
    \begin{tikzpicture}[memory/.style={rectangle,draw,very thick,anchor=east,font=\footnotesize},
    op/.style={rectangle,very thick,font=\scriptsize}]
    \node (mem) [memory] {7};
    \node (i) [above of=mem, yshift=-5mm,color=red!30!black!70!]
    {\large i};
    \node (t1) [right of=mem,xshift=\dist] {{\em Thread} $1$};
    \node (t1op1) [below of=t1,op,yshift=1mm] {{\tt bloqueia(i)}};
    \node (t1op2) [below of=t1op1,op] {sucesso: realiza bloqueio em
      {\tt i}!!};
    \node (t1op3) [below of=t1op2,op] {{\tt le(i) $\wedge$ incrementa(i)
      $\wedge$ escreve(i)}};
    \node (t1op4) [below of=t1op3,op] {{\tt libera(i)}};
    \node (t1op5) [below of=t1op4,op] {--};
    \node (t1op6) [below of=t1op5,op] {--};
    \node (t1op7) [below of=t1op6,op] {--};
    \node (t1op8) [below of=t1op7,op] {--};


    \node (t2) [right of=t1, xshift=3*\dist] {{\em Thread} $2$};
    \node (t2op1) [below of=t2,op] {{\tt bloqueia(i)}};
    \node (t2op2) [below of=t2op1,op] {{\color{red} falha! esperando...}};
    \node (t2op3) [below of=t2op2,op] {{\color{red} esperando...}};;
    \node (t2op4) [below of=t2op3,op] {{\color{red} esperando...}};
    \node (t2op5) [below of=t2op4,op] {sucesso: realiza bloqueio em
      {\tt i}!!};
    \node (t2op6) [below of=t2op5,op] {{\tt le(i) $\wedge$ incrementa(i)
      $\wedge$ escreve(i)}};
    \node (t2op7) [below of=t2op6,op] {libera {\tt i}};

    \path (t1) edge[->] (t1op1);
    \path (t1op1) edge[->] (t1op2);
    \path (t1op2) edge[->] (t1op3);
    \path (t1op3) edge[->] (t1op4);
    \path (t1op4) edge[->] (t1op5);
    \path (t1op5) edge[->] (t1op6);
    \path (t1op6) edge[->] (t1op7);

    \path (t2) edge[->] (t2op1);
    \path (t2op1) edge[->] (t2op2);
    \path (t2op2) edge[->] (t2op3);
    \path (t2op3) edge[->] (t2op4);
    \path (t2op4) edge[->] (t2op5);
    \path (t2op5) edge[->] (t2op6);
    \path (t2op6) edge[->] (t2op7);
  \end{tikzpicture}
\end{frame}

\begin{frame}{Causas de concorrência}
  \begin{description}
  \item[Interrupções]: uma variável pode ter seu valor alterado no
    registrador, porém ainda não atualizado na memória no momento
    da interrupção;
  \item[Preempção]: a mesma ação do item anterior, porém quem interrompe
    o fluxo de execução é o escalonador;
  \item[Multiprocessamento]: a qualquer momento uma variável pode ter
    seu valor alterado por uma CPU pode, ao mesmo tempo que em outra
    CPU a mesma variável é lida ou alterada.
  \end{description}
\end{frame}

\lecture{Mecanismos de sincronização}{sync:interface}

\subsection{\insertlecture}

\begin{frame}[fragile]
  \frametitle{{\em Spin lock}} \small O \alert{\em spin lock} bloqueia
  uma região do código fazendo com que outras {\em threads}
  fiquem esperando em {\em loop} até a saída do código da região crítica.\\
  \bigskip

  \begin{columns}
\begin{column}{.4\textwidth}
\begin{lstlisting}[framexleftmargin=5mm, frame=shadowbox,
    rulesepcolor=\color{red}]
  void lock(int *spin) {
    for (;;) {
      if (*spin == 0) {
        *spin = 1;
        break;
      }
    }
  }
  void unlock(int *spin) {
    *spin = 0;
  }
\end{lstlisting}
\end{column}
\begin{column}<2>{.4\textwidth}
  \begin{lstlisting}[framexleftmargin=5mm, frame=shadowbox,
    rulesepcolor=\color{blue}]
  // variável global
  struct {
    int id;
    // ...
    int spin;
  } proc;
  // ...
  // usando o bloqueio  
  lock(&proc->spin);
  // região crítica
  proc->id = 12345;
  unlock(&proc->spin);
\end{lstlisting}
\end{column}
  \end{columns}
\end{frame}

\lstset{morecomment=[s][\color{red}]{/*+}{*/}}
\begin{frame}[fragile]
  \frametitle{{\em Spin lock}} \framesubtitle{{\em Test and set}}
  \small Duas ou mais {\em threads} podem executar ao mesmo tempo em
  CPUs diferentes as linhas 4 e 5, obtendo o bloqueio.
  \only<2>{{\color{blue} A solução é usar a instrução de CPU {\tt
      test and set} que executa as operações 4 e 5 sem interrupção.}}
  
  \bigskip

  \begin{columns}
\begin{column}{.45\textwidth}
\begin{lstlisting}[framexleftmargin=5mm, frame=shadowbox,
    rulesepcolor=\color{red}]
  void lock(int *spin) {
    for (;;) {
      /*+ início: concorrência */
      if (*spin == 0) {
        *spin = 1;
      /*+ fim: concorrência */        
        break;
      }
    }
  }
\end{lstlisting}
\end{column}
\begin{column}<2>{.45\textwidth}
\begin{lstlisting}[framexleftmargin=5mm, frame=shadowbox,
    rulesepcolor=\color{blue},basicstyle={\scriptsize\color{blue}}]
  // comportamento da instrução
  // de hardware test and set
  int test_and_set(int *m) {
    int old = *m;
    *m = 1;
    return old;
  }
  
  void lock(int *spin) {
    for (;;) {
      if (test_and_set(&spin) == 0)
        break;
    }
  }
\end{lstlisting}
\end{column}
  \end{columns}
\end{frame}

\begin{frame}[fragile]
  \frametitle{Semáforo}
  \small O \alert{\em semáforo} foi proposto por Dijkstra~(1965)
  e usa um contador para assegurar que somente um fluxo de execução
  atinja a região crítica.
  \bigskip

  \begin{columns}
\begin{column}{.5\textwidth}
  \begin{lstlisting}[framexleftmargin=5mm,frame=shadowbox,
    rulesepcolor=\color{red}]
  void lock(int *sema) { // P()
    while (*sema < 0)  ;
    *sema--;
  }
  void unlock(int *sema) { // V()
    *sema++;
  }
\end{lstlisting}
\end{column}
\begin{column}<2>{.4\textwidth}
  \begin{lstlisting}[framexleftmargin=5mm, frame=shadowbox,
    rulesepcolor=\color{blue}]
  // variável global
  struct {
    int id;
    // ...
    int sema;
  } proc;
  // ...
  // usando o bloqueio  
  lock(&proc->sema);
  // região crítica
  proc->id = 12345;
  unlock(&proc->sema);
\end{lstlisting}
\end{column}
  \end{columns}
\end{frame}

%%%%%%%%%%%%%%%%%%%%%%%%%%%%%%%%%%%%%%%%%%%%%%%%%%%%%%%%%%%%%%%%
%% BUFFER
%%%%%%%%%%%%%%%%%%%%%%%%%%%%%%%%%%%%%%%%%%%%%%%%%%%%%%%%%%%%%%%%

\ifnum1=2

\lecture{Implementação de sincronização}{sync:implementation}

\section{\insertlecture}

\def\thetitle{\em Mutex}
\section{\thetitle}

\begin{frame}[fragile]{Exclusão mútua -- mutex}{\em Mutual exclusion}
  \small
  Tipo especial de semáforo onde somente uma tarefa é permitida na
  fila, ou seja, realiza a exclusão mútua entre $2$ tarefas como o
  \alert{\em spin lock}, porém, as tarefas ficam \alert{adormecidas} ao invés de
  ficarem esperando a liberação do bloqueio.\\
\bigskip

Fragmento de código C mostrando o uso da \alert{\em mutex} no SO Linux:

  \begin{lstlisting}[language=c,frame=single]
    mutex_init(&mutex);

    mutex_lock(&mr_lock);

    /* regiao critica */

    mutex_unlock(&mr_lock);
  \end{lstlisting}
  
\end{frame}

\def\thetitle{Semáforos}
\section{\thetitle}
\begin{frame}{\thetitle}
\only<1>{
  Mecanismo de sincronização:
  \begin{itemize}
  \item Criado pelo matemático e programador Edsger W. Dijkstra em 1965;
  \item Tipo abstrato de dados composto por um valor inteiro e uma
    fila de processos;
  \item Somente 2 operações são permitidas pelos semáforos {\bf P} (do
    holandês {\em proberem}, testar) e {\bf V} (do holandês {\em
      verhogen}, incrementar).
  \end{itemize}
  }

  \only<2>{ Quando uma tarefa tenta adquirir um bloqueio de uma região
    crítica usando \alert{semáforo}, um contador é decrementado. Se o
    valor do contador for negativo, a tarefa é colocada em uma fila de
    espera. A tarefa ao invés de ficar esperando, como acontece com o
    bloqueio ({\em spin-lock}), é colocada para dormir.  \bigskip }
  \begin{tabbing}
    {\bf P(S):}\= \\
    \> {\tt S.valor} $\leftarrow$ {\tt S.valor} $-$ $1$\\
    \> {\ se} {\tt S.}\= {\tt valor} $< 0$ \\
    \>\> {\tt então bloqueia(T) $\wedge$ insere(T, S.fila)}\\
    {\bf V(S):}\\
    \> {\tt S.valor} $\leftarrow$ {\tt S.valor} $+$ $1$\\
    \> {\tt se} {\tt S.}\= {\tt fila} $\neq$ $0$ \\
    \>\> {\tt então retira(T, S.fila) $\wedge$ acorda(T)}\\
  \end{tabbing}

\end{frame}


\hspace{-1cm}\begin{frame}[fragile]{Algoritmo de Peterson}{Gary L.\ Peterson, 1983}
  \lstset{language=C,
    commentstyle=\tiny\color{gray},
    columns=flexible,
    basicstyle=\scriptsize,
    emph={interessado},emphstyle=\color{red},
    emph=[2]{vez},emphstyle=[2]\color{blue}
  }
 
  \begin{lstlisting}
    #define FALSE 0
    #define TRUE 1
    #define N 2     /* no. de processos */

    int interessado[N];
    int vez;

    void peterson(int pid) /* argumento: processo que deseja entrar na região crítica */
    { 
          int outro = 1 - pid;          /* variável local: no. do outro processo */
          interessado[pid] = TRUE;  /* mostra interesse do processo atual no recurso */
          vez = pid;                /* altera o valor da vez */ 
      
          while (vez==pid && interessado[outro]==TRUE) 
          { /* espera ocupada */;}
      
          /* região crítica: acesso a recurso compartilhado */

          interessado[pid] = FALSE; /* indica saída de pid da região crítica */
    }

 \end{lstlisting}
 
\end{frame}

% \begin{frame}[fragile]{Exemplo de uso com pthread}
%   \lstinputlisting[firstline=26]{\srcdir/peterson.c}
% \end{frame}


\fi

%%% Local Variables:
%%% mode: latex
%%% TeX-master: main
%%% End:

