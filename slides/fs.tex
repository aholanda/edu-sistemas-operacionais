
\title{Sistema de arquivos}
\frame{\maketitle}

\begin{frame}{\inserttitle}{Definição}
  \small
  Interface entre os processos e os meios de armazenamento que fornece
  as abstrações e operações para a manipulação de bytes.
  
  \bigskip
  Dentre as principais abstrações encontram-se \alert{arquivo} e
  \alert{diretório}.\bigskip
  
  \pause
  Dentre as principais operações encontram-se:
  \begin{columns}
    \begin{column}{.4\textwidth}
      \begin{itemize}
      \item criar arquivo
      \item remover arquivo
      \item abrir um arquivo existente
      \item ler um arquivo aberto
      \end{itemize}
    \end{column}
    \begin{column}{.6\textwidth}
      \begin{itemize}
      \item gravar em um arquivo aberto
      \item fechar um arquivo aberto
      \item obter metadados de um arquivo
      \item modificar metadados de um arquivo
      \end{itemize}
    \end{column}
  \end{columns}

\end{frame}


\begin{frame}{\inserttitle}{Hierarquia}
  
  O sistema de arquivos normalmente obedece uma \alert{estrutura
    hierárquica} (árvore n-ária) específica para armazenamento de
  bytes.

\bigskip
\begin{center}
  \begin{tikzpicture}
    \tikzset{every node/.style={font=\scriptsize},
      file/.style={minimum height=1cm,blue,draw},
      directory/.style={semicircle,green!50!black,draw}}
    \node[red] {\em raiz}
    [edge from parent fork down]
    child {node[file] {arquivo}}
    child {node[directory] {diretório}
      child {node[file] {arquivo}}
      child {node[file] {arquivo}}
    };
\end{tikzpicture}
\end{center}

\end{frame}

